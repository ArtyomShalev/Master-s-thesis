%% Преамбула TeX-файла

% 1. Стиль и язык

\documentclass[utf8x]{G7-32} % Стиль (по умолчанию будет 14pt)
\usepackage[T2A]{fontenc}
\usepackage[russian]{babel}
\usepackage{svg}
\usepackage{cite}
\usepackage{pifont}
%\usepackage[demo]{graphicx}
\usepackage{caption}
\usepackage{subcaption}
\usepackage{comment}
% Остальные стандартные настройки убраны в preamble.inc.tex.
\sloppy

% Настройки стиля ГОСТ 7-32
% Для начала определяем, хотим мы или нет, чтобы рисунки и таблицы нумеровались в пределах раздела, или нам нужна сквозная нумерация.
\EqInChapter % формулы будут нумероваться в пределах раздела
\TableInChapter % таблицы будут нумероваться в пределах раздела
\PicInChapter % рисунки будут нумероваться в пределах раздела

% Добавляем гипертекстовое оглавление в PDF
\usepackage[
bookmarks=true, colorlinks=true, unicode=true,
urlcolor=black,linkcolor=black, anchorcolor=black,
citecolor=black, menucolor=black, filecolor=black,
]{hyperref}

% Изменение начертания шрифта --- после чего выглядит таймсоподобно.
% apt-get install scalable-cyrfonts-tex

%\IfFileExists{cyrtimespatched.sty}
 %   {
  %      \usepackage{cyrtimespatched}
   % }
    {
        % А если Times нету, то будет CM...
    }

\usepackage{graphicx}   % Пакет для включения рисунков

% С такими оно полями оно работает по-умолчанию:
 %\RequirePackage[left=30mm,right=20mm,top=20mm,bottom=20mm,headsep=0pt]{geometry}
% Если вас тошнит от поля в 10мм --- увеличивайте до 20-ти, ну и про переплёт не забывайте:
\geometry{right=20mm}
\geometry{left=30mm}


% Пакет Tikz
\usepackage{tikz}
\usetikzlibrary{arrows,positioning,shadows}

% Произвольная нумерация списков.
\usepackage{enumerate}

% ячейки в несколько строчек
\usepackage{multirow}

% itemize внутри tabular
\usepackage{paralist,array}
\setcounter{page}{7}

\DeclareCaptionLabelFormat{gostfigure}{Figure #2}
\DeclareCaptionLabelFormat{gosttable}{Table #2}
\DeclareCaptionLabelSeparator{gost}{~---~}
\captionsetup{labelsep=gost}
\captionsetup*[figure]{labelformat=gostfigure}
\captionsetup*[table]{labelformat=gosttable}
\usepackage{bm, braket}
%\newcommand{\ve}[1]{\bm{\mathbf{#1}}
\newcommand{\ve}[1]{\bm{#1}}
\newcommand{\bt}{BaTiO$_3$~}
\newcommand{\ga}{GaAs~}
%\usepackage{times}
%\usepackage{fontspec}
%\setmainfont{Times New Roman}
%\newfontfamily{\cyrillicfont}{Times New Roman}
\DeclareCaptionLabelFormat{gostfigure}{Рисунок #2}
\DeclareCaptionLabelFormat{gosttable}{Таблица #2}

% Настройки листингов.
\include{listings.inc}

% Полезные макросы листингов.
\include{macros.inc}

\begin{document}
\frontmatter % выключает нумерацию ВСЕГО; здесь начинаются ненумерованные главы: реферат, введение, глоссарий, сокращения и прочее.

% Команды \breakingbeforechapters и \nonbreakingbeforechapters
% управляют разрывом страницы перед главами.
% По-умолчанию страница разрывается.

% \nobreakingbeforechapters
% \breakingbeforechapters

%\include{00-abstract}

\tableofcontents

%\include{10-defines}
%\include{11-abbrev}

\chapter{\textbf{Введение}} 

С начала 21 века огромной популярностью пользовались объемные искусственные материалы - метаматериалы. Такие материалы позволяют реализовать электромагнитные свойства вещества, которые отсутствуют в природе. Однако из-за высоких потерь в материале, высокой дисперсии и сложности в изготовлении в микро- и наномасштабах было решено перейти в планарным метаматериалам - метаповерхностям. Метаповехрности относительно просты в изготовлении за счет современных методов литографии и 3D-печати, а субволновая толщина метаповерхности значительно снижает потери~\cite{2016_hsu_bic_review}.

Также с 2013 года экспериментально были обнаружены связанные состояния в континууме в двумерных периодических структурах, в частности в планарных фотонных кристаллах ~\cite{2013_hsu_nature}, которые открыли новое научное направление в фотонике, связанное с объяснением физики ССК и их применений. 

График по количеству публикаций и примеры использования.
(также график и знаковые работы)

Современное качественное научное исследованием невозможно представить без проведения численного моделирования. В связи с этим возникает острая необходимость проводить численный расчет подобных двумерных периоидических структур. Как правило для таких задач используют прямые численные методы, которые лежат в основе работы современных пакетов электромагнитного (ЭМ) моделирования. Такой подход остается самым распространенным в силу своей универсальности, однако он не лишен недостатков. В качестве альтернативы предлагается использовать полуаналитические методы, например метод РГУ, и еще методы решения подобных задач. Основным преимуществом полуаналитических методов, конечно же, является скорость расчета. 

Одним из способов проектирования новых метаповерхностей является решением обратной задачи. При таком методе по заданному физическому отклику системы (фронт фаза интенсивность расширить) подбирается оптимальная конфигурация как отдельных рассеивателей, так и всей системы в целом. Одним из эффективных современных подходов к решению обратных задач является применение различных алгоритмов машинного обучения. Для обучения моделей необходимы (многомиллионные или сколько там) выборки, в связи с чем к скорости решения прямых задач и, соответсвенно, к выбору инструментов нужно подходить очень внимательно.

Краткий обзор глав, что будет сказано. 


Тем не менее возможность моделирования двумерной периодической структуры с определенным мультипольным откликом может приводить к обнаружению новых эффектов, что будет описано в главе, посвященной






\mainmatter % это включает нумерацию глав и секций в документе ниже
\chapter{\textbf{Численные методы решения задач рассеяния света на двумерных периодических структурах}}
\label{1chapter}

\section{Численные методы решения уравнений Максвелла}
\label{sec:num_Maxwell}
Для решения любой ЭМ задачи можно воспользоваться прямыми методами решения уравнений Максвелла. Достоинством такого подхода является универсальность, однако эти методы практически всегда будут медленными. Существует множество разновидностей численных методов, которые используются в современных программных пакетах ЭМ моделирования. Однако чаще всего они будут являться производными от методов, описанных ниже: 

\begin{itemize}
	\item методы решения ур.Максвелла в дифференциальной форме, основанные на конечных разностях. Конечные разности составляются для временной (\textbf{FDTD} -- Finite Difference Time Domain) и частотных (\textbf{FDFD}~--~Finite Difference Frequency Domain) областей;
	\item метод конечных элементов (\textbf{FEM}~--~Finite Elements Method);
	\item метод граничного элемента, который зачастую упоминается как метод моментов (\textbf{MoM}~--~Method of Moments). 
\end{itemize}

FDTD является одним из самых распространенных методов решения электродинамических задач. Данный метод требует несколько операций (6 в трехмерном случае -- для каждой компоненты электрического и магнитного полей) на один узел сетки пространственно-временной сетки, поэтому этот метод крайне неэффективен в случае больших расчетных областях. Также недостатком данного метода является то, что он работает только со структурированными сетками (рис \ref{fig:test}).

FEM лишен данного недостатка и разбивает расчетную область на неструктурированную сетку (треугольную для двумерных задач и тетраедральную для трехмерных) (рис \ref{fig:test}), но данный метод значительно медленнее при расчетах во временной области. FEM обычно используется для вычисления гармонических полей и является стандартным методом для расчета вихревых токов.

\begin{figure}[h]
	\centering
	\begin{subfigure}{.5\textwidth}
		\centering
		\includegraphics[width=.7\linewidth, height=5 cm]{images/regular_grid.jpg}
		\caption{структурированная сетка}
		\label{fig:sub1}
	\end{subfigure}%
	\begin{subfigure}{.5\textwidth}
		\centering
		\includegraphics[width=.7\linewidth, height=5 cm]{images/irregular_grid.jpg}
		\caption{неструктурированная сетка}
		\label{fig:sub2}
	\end{subfigure}
	\caption{Виды сеток в различных численных методах}
	\label{fig:test}
\end{figure}

MoM работает с уравнениями Максвелла в интегральной форме, в качестве неизвестных выступают сторонние источники. Такой метод лучше подходит для открытых расчетных областей и небольших токопроводящих поверхностей.

Альтернативным подходом является использование полуаналитических методов. Например, для решения задачи рассеяния ЭМ волны на сферической частице используется теория Ми, которая позволяет получить точное аналитическое решение. 


\section{Основные сведения из теории Ми}
\label{sec:Mie_theory} 

Теория Ми рассматривает рассеяние ЭМ волны сферической частицей, такое рассеяние также называют рассеянием Ми. Данная задача является классической задачей электродинамики, которая была решена аналитически Густавом Ми в 1908 году~\cite{Mie}. Теория Ми является общей и приводит к различным частным случаям при рассмотрении  определенного отношения диаметра частицы к длине волны:

\begin{itemize}
	\item в случае, когда диаметр частицы много меньше длины волны $d << \lambda$, происходит Рэлеевское рассеяние. При данном рассеянии длина волны света не изменяется, такое рассеяние также называется упругим. Внешнее ЭМ поле поляризует частицу, возбуждая в ней переменный дипольный момент. Такой дипольный момент переизлучает свет с частотой колебания диполя и дипольной диаграммой направленности;
	\item в случае, когда $d \approx \lambda$ или чуть меньше (частица становится субволновой), диаграмма направленности становится более сложной в силу возникновения интерференции волн, отраженных от различных участков поверхности сферы. Такой случай называется резонансным, так как в рассеивающей сфере может укладываться несколько длин волн. При этом возможны ситуации, когда в рассеянии значительно доминирует вклад какой-то определенной векторной сферической гармоники (ВСГ), математический смысл которой будет рассмотрен далее. Тогда на больших расстояниях от частицы диаграмма направленности рассеянного поля будет похожа на соответствующую диаграмму направленности угловой части ВСГ. 
	\item в случае, когда $d >> \lambda$, поверхность сферы будет в первом приближении вести себя как плоская поверхность, и рассеяние света можно в терминах отражения и преломления в соответствии с формулами Френеля.  
\end{itemize}  

Математика в теории Ми относительно проста, однако местами громоздка. Представить разложение ЭМ полей во всем пространстве (особенно с высокой точностью) в виде рядов  до появления вычислительной техники было поистине трудоемкой задачей. Однако с развитием электронно-вычислительной техники строгое решение задачи Ми стало значительно проще. Преимуществами использования теории Ми являются сокращенный объем вычислений, что значительно увеличивает скорость расчета, и возможность оценивать вклады различных мультиполей в общую картину рассеяния. Ниже будет кратко рассмотрен ход решения задачи рассеяния плоской волны сферической частицей. 


Для решения векторных волновых уравнений удобно перейти от векторной функции к скалярной. В сферической системе координат можно ввести вспомогательные векторные функции которые будут связаны со скалярной функцией следующим образом:

\begin{equation*}
\mathbf{M} = \nabla \times (\mathbf{c}\psi)
\end{equation*}

\begin{equation*}
\mathbf{N} = \frac{\nabla \times \mathbf{M}}{k} 
\end{equation*}

где $\psi$ -- скалярная функция, которая является решением скалярного волнового уравнения, $\mathbf{c}$ -- постоянный вектор, а $k$ -- волновое число.

Данные функции обладают всеми необходимыми свойствами ЭМ поля -- они удовляетворяют векторному волновому уравнению, их дивергенции равны нулю, а также ротор $\mathbf{M}$ пропорционален $\mathbf{N}$ и наоборот.
Таким образом, задача нахождения решений волнового уравнения для векторного поля сводится к задаче решения скалярного волнового уравнения. Функции $\mathbf{M}$ и $\mathbf{N}$ называются магнитными и электрическими ВСГ соответственно.

В ходе решения скалярного волнового уравнения, который подробно описан в книге~\cite{Bohren1998} получается, что функция $\psi$ разделяется на четную и нечетную и имеет вид:
\begin{equation} \label{eq:psi1}
	\psi_{emn} = \mathrm{cos}m\varphi P_n^m(\mathrm{cos}\theta)z_n(kr) 
\end{equation}
\begin{equation} \label{eq:psi2}
	\psi_{omn} = \mathrm{sin}m\varphi P_n^m(\mathrm{cos}\theta)z_n(kr) 
\end{equation}

где $z_n$ -- любая из четырех сферических функций Бесселя.

В таком случае после применений соотношений (\eqref{eq:psi1} -- \eqref{eq:psi2}) ВСГ также разделяются на четные и нечетные $\mathbf{M}_{o(e)mn}$ и $\mathbf{N}_{o(e)mn}$. В силу своей громоздкости явный вид ВСГ приведен в приложении (ссылка на него). Стоит отметить, что ВСГ являются функциями волнового числа $k$ и радиус-вектора $\mathbf{r}$, а также то, что верхний индекс ВСГ указывает на то, какая сферическая функция Бесселя будет использована при расчете ВСГ.

После этого можно разложить любое поле по ВСГ. В модельных задачах падающее поле обычно представляется в виде плоской волны, и ее разложение по ВСГ имеет вид:
\begin{equation*}
E_{inc} = E_0e^{ikrcos\theta}\mathbf{e}_x = E_0\sum_{n=1}^{\infty} i^n \frac{2n+1}{n(n+1)}(\mathbf{M}_{o1n}^{(1)}-i\mathbf{N}_{e1n}^{(1)})
\end{equation*}

\begin{equation*}
H_{inc} = \frac{-k}{\omega\mu}E_0\sum_{n=1}^{\infty} i^n \frac{2n+1}{n(n+1)}(\mathbf{M}_{e1n}^{(1)}+i\mathbf{N}_{o1n}^{(1)})
\end{equation*}

Рассеянные поля имеют вид:
\begin{equation*}
 	E_{s}= \sum_{n=1}^{\infty} E_n (ia_n\mathbf{N}_{e1n}^{(3)} - b_n\mathbf{M}_{o1n}^{(3)}) 
\end{equation*}

\begin{equation*}
	H_{s} = \frac{k}{\omega\mu}\sum_{n=1}^{\infty} E_n (a_n\mathbf{M}_{e1n}^{(3)} + ib_n\mathbf{N}_{o1n}^{(3)}) 
\end{equation*}

где $E_n = \frac{iE_0(2n+1)}{n(n+1)}$, а верхний индекс $(3)$ у ВСГ показывает то, что радиальная зависимость функций (\eqref{eq:psi1} -- \eqref{eq:psi2}) используются сферические функции Ханкеля.

Поля внутри сферической частицы имеют вид:
\begin{equation*}
	E_{i}= \sum_{n=1}^{\infty} E_n (-id_n\mathbf{N}_{e1n}^{(1)} + c_n\mathbf{M}_{o1n}^{(1)}) 
\end{equation*}

\begin{equation*}
	H_{i} = \frac{-k_1}{\omega\mu_1}\sum_{n=1}^{\infty} E_n (d_n\mathbf{M}_{e1n}^{(1)} + ic_n\mathbf{N}_{o1n}^{(1)}) 
\end{equation*}

После применения граничных условий получаются выражения для частотной зависимости коэффициентов $c_n(\omega)$, $d_n(\omega)$, $b_n(\omega)$ и $a_n(\omega)$.
В приложении (ссылка на него) приведены полные выражения для коэффициентов разложения полей, которые получаются после применения граничных условий.

\section{Дополнение теории Ми на случай несферических частиц. Метод Т-матриц}
\label{sec:T_matrix}
Опираясь на основные положения теории Ми, можно перейти к рассмотрению метода Т-матриц, который является дополнением теории Ми на случай несферических частиц, которые повсеместно встречаются как в природе, так и в искуственных рассеивателях с произвольной конфигурацией, придуманной человеком. Чисто аналитическое решение задач рассеяния ЭМ поля на частицах такой формы невозможно, в связи с чем возникает потребность в создании высокоэффективных численных методов решения таких задач. 

В формализме Т-матриц ЭМ волны также как и в теории Ми раскладываются на ВСГ. Вся информация о мультипольном отклике частицы содержится в ее Т-матрице. Т-матрица связывает коэффициенты разложения электрического и магнитного рассеяннного и падающего полей следующим образом:

\begin{equation} \label{Tmatrix1}
	\begin{pmatrix}	b_{e}  \\
		b_{h}   \end{pmatrix} = 
	T\begin{pmatrix}	a_{e}  \\
		a_{h}   \end{pmatrix}
\end{equation}

В общем случае в Т-матрице также содержатся члены, описывающие влияние падающего электрического поля на рассеянное магнитное и наоборот:
\begin{equation} \label{Tmatrix2}
		Т = 
		\begin{pmatrix}	
			T_{ee} & T_{eh}  \\
			T_{he} & T_{hh}  
		\end{pmatrix}
\end{equation}

А матрица $T_{ee}$ имеет вид:
\begin{equation} \label{Tmatrix3}
	T_{ee} = \begin{pmatrix}
		T_{ee}^{11} & T_{ee}^{12} & ... \\
		T_{ee}^{21} & T_{ee}^{22} & ... \\
		...			&	...		  & ... \\
	\end{pmatrix}
\end{equation}	


Т-матрица не зависит от типа падающего поля и ориентации частицы в пространстве. Т-матрица зависит только от размера частицы, ее материальных параметров, формы и длины волны света. Можно сказать, что Т-матрица содержит полную информацию о способности частицы рассеивать падающее на нее ЭМ поле разной частоты. 

Это приводит как минимум к двум весомым преимуществам при использовании формализма Т-матриц при решении задач рассения:
\begin{itemize}
	\item если Т-матриц расчитана для одной ориентации частицы, то для другой ориентации (или направления падающего поля) ее достаточно пересчитать аналитически;
	\item Т-матрицы используются при моделировании систем, состоящих из множества случайно ориентированных частиц, так как с их помощью можно получить выражения в замкнутой форме.  
\end{itemize}


Большую часть расчетного времени в данном методе занимает расчет большого числа поверхностных интегралов, которые требуют расчетов функций Бесселя и ВСГ. Стоит отметить, что любые численные методы решения ЭМ задач, которые лежат в основе современных программных пакетов, способны посчитать Т-матрицу. Однако методы, рассмотренные ниже дают полуаналитическое решение.

Аналитические выражения для Т-матриц для сфер можно получить из теории Ми. Однако для расчета Т-матриц для несферических частиц существует ряд методов, к которым относят метод расширенных граничных условий (\textbf{EBCM} -- extended boundary condition method) и дискретное дипольное приближение (\textbf{DDA} -- discrete dipole approximation), которые подробно рассмотрены в работах~\cite{2015Kahnert}~\cite{2014TMatrixComparison} 

\begin{comment}
\subsection{DDA} 
Вообще надо смотреть 25 ссылку и как там этот метод используется для вычисления Т-матриц

По сравнению с FDTD метод DDA решает уравнения Максвелла в частотной области. Рассмотрение данного метода начинается с уравнений:

\begin{equation*}
	\nabla \times \mathbf{H} + i\omega\varepsilon\mathbf{E} = 0
\end{equation*}

\begin{equation*}
	\nabla \times \mathbf{E} - i\omega\mu\mathbf{H} = 0
\end{equation*}

формулы выше скорее всего не нужны

Рассмотрим лучай диэлектрической частицы в неоднородной среде:
\begin{equation*}
	\nabla \time \nabla \mathbb{E(r)} - k_0^2\mathbf{E(r)} = i\omega\mathbf{j(r)}
\end{equation*}

итд расшифровки

Решение - функци Грина

В итоге решение волнового уравнения (ссылка)


пояснение к формуле

Уравнение (ссылка) также называется объемным интегральным уравнением. Данное уравнение можно решать разными способами (МОМ может быть), но в данном случае будет рассмотрен DDA, который был впервые предложен в работе (14 ссылка из ревью).

Основная идея ДДА заключается в разбиении разбить объем частицы на множество маленьких объемов. Это можно сделать по-разному, но самый простой и очевидный способ -- кубическая сетка.
РИСУНОК можно 

Из рисунка также видно, что по сравнению с методом FDTD дискретизации подвергается только сам объем частицы.
После дискретизации уравнение (ссылка) переходит в СЛАУ, которая решается стандартными методами.

можно переписать интеграл в виде суммы но лучше не стоит

Дискретизация должна быть такой, чтобы каждый элементарный объем был много меньше длины волны. В таком случае можно пренебречь любыми неравномерными распределениями амплитуды поля по элементарному объему. С физической точки зрения это значит, что все заряды в элементарном объеме осциллируют в фазе

В таком случае можно представить, что в каждом элементарном объеме колеблются одиночные диполи. Тогда вся структура описывается массивом одиночных диполей, расстояние между которыми определяется шагом сетки.

Если опустить промежуточный вывод

то приходим к СЛАУ, которое связывает какое то (наведенное виимо) поле и падающее поле в виде:
\begin{equation*}
	\mathbf{E_inc^m} = \sum_{k=1}^{M} \mathbf{Q_{mn}} \cdot \mathbf{E_{exc}^k}
\end{equation*}

СЛАУ решается стандартными методами

Чтобы определить поле в произвольной точке вне частицы, то нужно воспользоваться интегральным уравнение (ссылка), в котором аргумент $\mathbf{r}$ будет принадлежать области вне частицы. А в источнике (в какой там части) будет учтено поле наводимое частицей. Плотность тока в каждом элементарном объеме можно выразить через наведенный дипольный момент $\mathbf{j_m} = \frac{\mathbf{p_m}\omega}{iV_m}$, также дипольный момент связан с наведенным частицей полем соотношением $\mathbf{p_m} = \mathbf{t_m} \cdot \mathbf{E_exc^m}$, где $\mathbf{t_m}$ матрица поляризуемости ячейки $m$. Тогда (интегральное уравнение ссылка) можно переписать в виде (61)

Для решения задачи рассеяния остается только подобрать такую модель поляризуемости, которая бы корректно описывала диэлектрические свойства частицы. 

Стоит отметить, что данный алгоритм не учитывает форму и характер распределения материальных параметров частицы. Это значит, что данный метод применим для произвольной формы частицы.    
\end{comment}

EBCM широко распространен и является одним из наиболее эффективных методов для расчета рассеяния ЭМ волн на частицах произвольной формы. Данный метод впервые был предложен Вотерманом ~\cite{1965Waterman} в цикле его работ в 1965-1979 годах. Данный метод был применен при решении множества электромагнитных задач (расчет диаграмм направленности для периодических структур, состоящих из рассеивателей различных форм, акустических и плазмонных задачах рассеяния и при моделировании многослойных структур) ~\cite{2004MischenkoDatabase}.
Несмотря на такую богатую историю использования и обширный спектр применений, численная реализация данного метода в некоторых случаях (например, для выпуклых рассеивателей с высоким аспектным отношением) остается до сих пор актуальной ~\cite{2011LeRu}.


\section{Двумерные решеточные суммы(еще в разработке)}
Изучение распространения волн в периодических структурах приводит к ряду интересных математических задач (каких?)
Особенностью решения подобных задач является то, что решение на бесконечной структуре можно свести к решению задачи в одной элементарной ячейке. Существует множество подходов решения подобных задач, одним из которых является использование функций Грина. Однако, если функции Грина для бесконечной области достаточна проста, то использование подхода функций Грина для одной элементарной ячейки приводит к появлению решеточных сумм вида $\sum_\Lambda \mathrm{exp}(i\bm{\beta} \bm{R_m})G_0(\bm{r},\bm{R_m})$, где $\Lambda$ -- периодический массив, $\bm{R_m}$ -- радиус-вектор, $\bm{\beta}$ -- постоянная распространения или в терминах периодических структур Блоховский вектор, а $G_0(\bm{r},\bm{r'})$ -- функция Грина, которая описывает влияние источника, находящегося в точке $\bm{r'}$ на значение в точке $\bm{r}$ . Произведение $\mathrm{exp}(i\bm{\beta} \bm{R_m})G_0(\bm{r},\bm{R_m})$ обозначают $G_\Lambda$ и называют квази-периодической функцией Грина. Данная функция описывает влияние периодического массива точечных источников с одинаковой амплитудой, но разным сдвигом фаз на поле в точке $\bm{r}$. При решении задач рассеяния ЭМ волн на периодических структурах важно, чтобы $G_\Lambda$ рассчитывалась быстро и точно.

Методы, основанные на вычислении $G_\Lambda$ для уравнения Гельмгольца, нашли себе применения во многих областях физики, например в дифракции медленных электронов (84 линтон), акустике(91 линтон), рассеяние ЭМ волн на экранах с отверстиями(63 линтон), распространение волн в композитных материалах(72 линтон), определении зонной структуры фотонных кристалов(96 линтон) и проектировании конфигурации метаповерхностей(115 линтон).

Есть три основные причины для изучения решеточных сумм:
\begin{itemize}
	\item задачи рассеяния света на периодических структурах часто могут быть сведены к системе линейных уравнений, в которых матрица коэффициентов будет включать решеточные суммы;
	\item в виде коэффициентов разложения ряда решеточные суммы могут быть частью эффективной численной схемы вычисления $G_\Lambda$, так как значения данной функции используется неоднократно;
	\item в начальном виде решеточные суммы представляют из себя члены условно сходящегося ряда. Для численной эффективной реализации вычисления членов такого ряда необходимо провести ряд преобразований, что, в свою очередь, является интересной задачей на стыке математики и компьютерных наук.
\end{itemize}  

Существуют разные подходы для преобразования условно сходящихся рядов в сходящиеся с высокой скоростью. В данной работе внимание будет уделено подходу, основанному на суммировании Эвальда, и численной реализации этого подхода Камбе (ссылки на место в работе)

это удалить скорее всего
С математической точки зрения решеточную сумму можно охарактеризовать как сумму вида $ $, где $\Lambda$ - массив из точек, каждая из которых характеризуеттся радиус-вектором $ $ , а $ $ - заданная функция. (стоит ли говорить про периодичность?). Такие суммы представляют интерес, когда $ $, где $ $ -- функция Грина для определенного уравнения в частных производных, описывающая эффект в точке $ $ от источника в точке $ $, а $ $ - сила источника. Решеточные суммы нашли широкое применение при исследовании электростатических взаимодействий в ионных кристаллах еще в 19 веке (ссылка из Линтона) 

В работе (ссылка на линтона) дается обширный обзор решеточных сумм, которые используются при решении уравнения Гельмгольца. В этом разделе хочется поподробнее остановиться на двумерных решеточных суммах в трехмерном пространстве, ведь именно их вычисление требуется для работы программы MULTEM, которая рассмотрена в главе (ссылка на главу).
  
Для понимания (решеточных сумм) необходимо ввести ряд понятий и ввести их математическое определение:
1) прямая и обратная решетки
2) формула суммирования Пуассона
3) условно сходящиеся ряды (под вопросом)
4) квази-периодическая ф Грина и ее интегральное представление


Вычисление $G_\Lambda$ является фундаментальным при решении задач рассеяния ЭМ волн. Для их вычислений написано множество алгоритмов (дать краткий обзор по ним). Далее будут рассмотрены некоторые представления КПФГ, необходимые для понимания решеточных сумм.
формулки для ФГ точечного источника

Однако данное решение справедливо только для вещественной К, в общем случае К должно быть комплексным. Более того, изучение вихревых решеток в сверхпроводниках требует, чтобы к был чисто мнимым (32 линтон)

Определение КПФГ 

Стоит отметить, что в случае двумерной решетки КПФГ безразмерна, хотя в трехмерном случае имеет размерность обратной длины.


5) двойные ряды

КПФГ представлена в виде суммы по прямой решетке. ФСП позволяет представить данную сумму в виде суммы по обратной решетке. Такие ряды называются двойными 

Аккуратно описать про соответсвие д и длам

Возможно всю эту конкретику во вторую главу, а тут только теорию.

Резюме: 
Для вычисления функции Грина (какой именно функции Грина - расширить ее название) используется реализация Гуерин-Енок-Тайеб (проверить правописание на русском). Альтернативный метод Камбе имеет точность вычисления близкой к машинной и в как минимум в 12 раз быстрее и сходится даже в плоскости решетки.

Альтернативный метод основан на суммировании Эвальда (2,3 мороз) и приводит к экспоненциальной сходимости представления(убрать функции Грина для свободного пространства $G_{0\Lambda}$ и ее решеточных сумм (решеточные суммы функции?). Такое представление было впервые предложено Камбе в 1967 (2,3 мороз)

(нужно ли давать громоздкие формулы?)

С тех пор данное представление является неотъемлемой частью программ для численного расчета, основанных на методе гриновских функций Корринги – Кона – Ростокера, предназначенного для определения зонной структуры периодических структур и теории дифракции медленных электронов, которая описывает дифракцию квантовых и классических волн от двумерных периодических структур в 3Д (ВООООТ??) Эти программы успешно себя показали при решении различных физических задач с 70-ых годов 20 века. Стоит отметить, что данные программные реализации могут успешно решать как скалярные, так и векторные уравнения Гельмгольца.

Реализация Камбе (ур 6-8) не были включены в анализ (1 источник мороз). Во-первых, при известной сигме и к парал идентичные решеточные суммы используются в (ур 4) для расчета $G_{0\Lambda}$ везде, например, в плоскости структуры ($z = 0$) (какое условие?) (для сравнения в источнике 1 ряды сходятся только при $|z|\geq 10^{-6}$)
Во-вторых, не составляет проблемы рассчитать функцию Грина с машинной точностью. В третьих, выражения (6-8) Камбе предоставляют unparallel скорость сходимости по сравнению с (1 ссылка).

После первого вычисления решеточной суммы $D_{lm}$ до заданного количества мультиполей дальнейшие вычисления функции Грина зависят только скорости вычисления функций Бесселя и сферических гармоник.

Таблица сравнения из мороза. Там есть и скорость сходимости суммирования Эвальда-Камбе

\subsection{Суммирование Эвальда} (1995 год суммирование Эвальда)



\chapter{\textbf{Технический обзор (название поменять)}}

\section{Введение}
MULTEM был впервые представлен в работе ~\cite{1998Stefanou} в 1998 году. Данная программа позволяет рассчитывать зонную структру и коэффициенты прохождения, отражения и поглощения бесконечных двумерных периодических структур. Изначально данная программа решала задачу расчета фотонных кристаллов, однако в ней не предусмотрено ограничений на отношение периода решетки к длине волны излучения. В связи с этим есть предпоссылки для использования данного инструмента при моделировании любых двумерных периодических структур, в то числе и метаповерхностей. MULTEM позволяет моделировать многослойные структуры, при этом слои могут состоять из разного материала. MULTEM может работать только с решетками Браве в каждом слое и только со сферическими рассеивателями. Несмотря на эти ограничения, он по-прежнему актуален, о чем свидетельтствует количество цитирований оригинальной работы за последние 10 лет (привести сколько цитирований по обеим публикациям), а с открытием такого явления в фотонике как связанные состояния в континууме (ССК) (ссылка на работу 2008 года) и их экспериментальным обнаружением в планарном фотонном кристалле ~\cite{2013_hsu_nature}, то для некоторых систем MULTEM может быть актуален даже в своей оригинальной реализации ~\cite{2019_zarina_multipolar_bic}.

\section{Принцип работы MULTEM} 

Программа MULTEM написана в процедурном стиле, поэтому удобно будет рассмотреть математические операции в соответствии с процедурами (или субрутинами на языке программирования Fortran) и функциями, которые их выполняют.

\subsection{Описание входных данных (возможно в приложение вместе с конфиг файлом)}
Чтение входных данных осуществляется через конфигурационный файл, содержимое которого представлено в приложении (ссылка на приложение).

Параметр KTYPE определяет в каком режиме будет производиться расчет. KTYPE принимает значение от 1 до 3. 1 -- направление падающей волны определяется двумя полярными углами $\theta$ и $\varphi$, 2 -- направление падающей волны определяется частотой волны и проекциями волнового вектора $k_x$ и $k_y$, 3 -- программа работает в режиме расчета зонной структуры.

Параметр KSCAN определяет тип сканирования в определенном диапазоне. 1 -- частотное сканирование, 2 -- сканирование по длинам волн.

Параметр KEMB определяет наличие среды вокруг структуры. 0 -- отсутствует, 1 -- присутствует. 

Параметр LMAX определяет максимальный порядок ВСГ, которые будут приняты в рассмотрение. Максимальное значение LMAX = 7.

Параметр NCOMP определяет количество компонентов в одном срезе многослойной структуры, а параметр NUNIT определяет количество таких срезов.

Параметры ALPHA, ALPHAP и FAB определяют геометрию решетки. ALPHA соответствует длине вектора трансляции решетки в одном из направлений и всегда равно 1. ALPHAP соответствует вектору трансляции в другом направлении и задается в количестве ALPHA. FAB определяет угол между двумя векторами трансляций решетки. Таким образом, в случае квадратной решетки ALPHA = ALPHAP = 1, FAB = 90. 

Переменная RMAX?

Параметры ZINF и ZSUP определяют минимальное и максимальное  значение частотного диапазона (или диапазона длин волн в зависимости от выбранного значения KSCAN). Значение частоты и длины волны нормируются на параметр ALPHA. NP определяет количество точек на спектр.

Параметры THETA/AK(1) и PHI/AK(2) определяют полярные углы или проекции волнового вектора на двумерную решетку в зависимости от выбранного значения KTYPE.

Параметр POLAR определяет поляризацию падающей волны и принимает значения S (TE -- электрическое поле параллельно плоскости падения волны) или P (TM -- магнитное поле параллельно плоскости падения волны).

Параметр FEIN определяет направление вектора поляризации (в градусах) и его можно задать только в случае нормального падения волны. 

Параметр IT относится к определению состава элементарного среза структуры. 1 -- однородная плоскость, 2 -- несколько слоев сферических рассеивателей. 

Параметры MUMED и EPSMED определяют комплексные значения материальных параметров среды, в которой моделируется решетка. 

Параметр NPLAN определяет количество плоскостей в одном слое, а NLAYER -- количество слоев.

Параметр S определяет радиус сферы. Данный параметр нормируется на ALPHA.

Параметры EPSSPH и MUSPH определяют комплексные значения материальных параметров сфер.

Параметры EPSEMBL, MUEMBL и EPSEMBR, MUEMBR имеют смысл только в случае моделирования однородной плоскости и определяют материальные параметры верхнего и нижнего полупространства соответственно.


\subsection{Структура программы и основные математические выражения (есть в ворде, нужно перебить в латех -- не успел пока сделать)}
\begin{comment}
Субрутина ПССЛАБ вычисляет матрицы прохождения и отражения для плоскости из сфер в однородной среде (уравнение 46). 
Матрица ЗЕТ (уравнения 27-30) вычисляется по алгоритму, основанному на суммировании Камбэ, которая подобно суммированию Эвальда разбивает сумму по двумерной решетке (ур 27) на две суммы (одна в прямом и другая в обратном пространстве (ур 17)), при этом расчете используется комлексная функция ошибок или функция Фадеевой (о ней будет позже). 

Константы из ур 29 вычисляются в субрутине ЭЛМГЕН и хранятся в массива ЕЛМ. Данные коэффициенты не зависят от решетки ли частоты и поэтому ЕЛМГЕН вызывается один раз в начале. 

ХМАТ за счет свойства 31 перераспределяет элементы матрицы З и переводит ее в блочно-диагональную форму. Матрица З разбивается на две матрицы в зависимости от четности сумм $l+m$ и $l'+m'$. Эти матрицы хранятся в массивах ХИВЕН и ХОДД

Задача, которую выполняет субритина ПССЛАБ заключается в решении СЛАУ 36, которые за счет свойства 32 матрицы омега разбиваются на две системы, состоящих из $l_{max}(l_{max}+2)$ уравнений.


Далее описание из статьи.

блок схему алгоритма привести
\end{comment}

\section{Модификации исходного кода (в разработке)}


\subsection{erf}





\chapter{\textbf{Сравнение CST Microwave Studio и MULTEM}}

\section{Введение}
Коммерческая основа CST Microwave Studio (далее CST) накладывает на нее определенные обязательства к точности и воспроизводимости результатов моделирования, а коммерческий успех данного продукта позволяет сделать вывод о том, что его результатам можно доверять.

Что же касается программ с открытым исходным кодом, к которым относится MUTLEM, то, очевидно, что разработчики редко могут гарантировать определенный результат, поэтому к использованию таких программ в научных или промышленных исследованиях стоит подходить с осторожностью. В связи с этим перед использованием MUTLEM в расчетах необходимо верифицировать правильность его работы. 

В данной главе будет проведно сравнение этих двух программных продуктов по скорости и точности выполняемых ими вычислений на примере двумерной бесконечной периодической структуры.

\section{Описание исследуемой системы}
В качестве исследуемой системы выступала двумерная бесконечная периодическая структура с квадратной решеткой. В каждый узел решетки помещалась диэлектрическая сфера с относительной диэлектрической проницаемостью $\varepsilon = 50$, диаметр сферы $d = 20$ мм, а период решетки $a = 40$ мм. Система облучалась плоской ТЕ поляризованной волной под углом $\theta=45^{\circ}$ к нормали структуры. Внешний вид исследуемой системы представлен на рисунке \ref{fig:system_concept}.

\begin{figure}[h]
	\begin{center}
		\includegraphics[width=0.9\textwidth]{images/system_concept.jpg}
		\caption{Внешний вид исследумой системы}
		\label{fig:system_concept}
	\end{center}
\end{figure}

\section{Аппроксимация спектра}
Спектры, полученные MULTEM и CST Microwave Studio были сняты в микроволновом диапазоне 0.13 -- 0.155 м. Спектр коэффициента отражения представлен на рисунке \ref{fig:fano_fit}. В качестве параметра, по которому производилось сравнение было выбрано положение резонанса $\lambda_0$.

Для извлечения $\lambda_0$ была проведена аппроксимация при помощи программного кода, написанного на языке программирования Python. Полученный спектр аппроксимировался формулой резонанса Фано, предложенной в работе ~\cite{FanoKK}. Резонанс Фано в зависимости от длины волны описывается формулой:
\begin{equation*}
	f(\lambda) =  \frac{I(q+\lambda - \lambda_0)^2}{1+q^2(1+(\lambda - \lambda_0)^2)}
\end{equation*}

Данная аппроксимация достаточно хорошо описывает спектр, полученный численным расчетом, что продемонстрировано на рисунке \ref{fig:fano_fit}.
\begin{figure}[h]
	\begin{center}
		\includegraphics[width=0.9\textwidth]{images/fano_fit.png}
		\caption{Аппроксимация спектра формулой резонанса Фано}
		\label{fig:fano_fit}
	\end{center}
\end{figure}
Процедура аппроксимации позволяет получить точное значение $\lambda_0$, что является необходимым для сравнения результатов моделирования разных программ.

\section{Моделирование в MUTLEM}

Конфигурационный файл, который был использован для задания параметров моделирования приведен (ссылка на приложение).
Параметр $l_{max}$ определяет максимальный порядок ВСГ, которые учитываются при расчете. Чем этот параметр выше, тем выше точность расчета, однако при этом значительно возрастает время расчета. Максимальное значение параметра $l_{max} = 7$, тем не менее для некоторых задач достаточно и меньших значений параметра, так как дальнейший вклад ВСГ в результаты расчета становится незначительным. На рисунке \ref{fig:lmax} представлена зависимость положения резонанса от всех значений $l_{max}$, которые позволяет выставить MULTEM.

\begin{figure}[h]
	\begin{center}
		\includegraphics[width=0.9\textwidth]{images/lmax.png}
		\caption{Зависимость положения резонанса от количества рассматриваемых ВСГ (рисунок подправить)}
		\label{fig:lmax}
	\end{center}
\end{figure}

При $l_{max} \geq 3$ видно, что дальнейшее изменение $\lambda_0 < 10^{-6}$ \%. В связи с этим было решено зафиксировать значение $l_{max} = 3$ и использовать его при дальнейшем сравнении с результатами расчета CST Microwave Studio. При $l_{max} = 3$ $\lambda_0 = 0.14274$ м.

\section{Моделирование в CST Microwave Studio}
При моделировании исследуемой системы в CST был использован расчет в частотной области (Frequency Domain Solver), так как данный способ позволяет считать все порты одновременно и независимо друг от друга и является отличным инструментом для моделирования периодических структур. Для моделирования двумерных бесконечных периодических структур используются специальные граничные условия "Unit cell"\,, которые продлевают элементарную ячейку в двух направлениях до бесконечности.

\begin{figure}[h]
	\begin{center}
		\includegraphics[width=0.9\textwidth, height = 7 cm]{images/cst_model.jpg}
		\caption{Исследуемая модель в CST}
		\label{fig:cst_model}
	\end{center}
\end{figure}

\begin{figure}
	\begin{center}
		\includegraphics[width=0.9\textwidth]{images/cst_vs_multem.png}
		\caption{cst vs multem}
		\label{fig:cst_vs_multem}
	\end{center}
\end{figure}

Плотность сетки регулировалась определением числа тетраедров на всю расчетную область. В CST плотность сетки задается во вкладке "Global properties" (рис. \ref{fig:mesh_config}) Определяется минимальное и максимальное количество ячеек на одну длину волны как для модели, так и для среды, в которую модель помещена. Также при расчетах использовалась адаптивная сетка, поэтому количество тетраедров, используемых в расчете выше, чем при ручном определении сетки (\ref{fig:mesh}). 

\begin{figure}[h]
	\centering
	\begin{subfigure}{.5\textwidth}
		\centering
		\includegraphics[width=.7\linewidth]{images/mesh_config.jpg}
		\caption{A subfigure}
		\label{fig:mesh_config}
	\end{subfigure}%
	\begin{subfigure}{.5\textwidth}
		\centering
		\includegraphics[width=.7\linewidth]{images/mesh.jpg}
		\caption{A subfigure}
		\label{fig:mesh}
	\end{subfigure}
	\caption{A figure with two subfigures}
	\label{fig:cst_mesh}
\end{figure}




На рисунке \ref{fig:cst_vs_multem} представлена зависимость относительной ошибки расчета $\lambda_0$, спектры отражения для которой были расчитаны при помощи CST, от числа тетраедров.

Данная зависимость демонстрирует, что результат расчета CST сходится к результату расчета MULTEM, так как рассчитанные точки в двойном логарифмическом масштабе укладываются на прямую с коэффициентом детерминации $R^2 = 0.9992$. Несмотря на то, что нельзя утвержать, что при дальнейшем увеличении количества тетраедров тренд также останется близким к линейному, данная зависимость демонстрирует, что при самом длительном расчете в CST ($7200$ с) относительная ошибка $\delta\lambda_0 = 3.2 \cdot 10^{-4}$ \%. В то время, как время расчета в MULTEM составило (столько то сек). В таблице \ref{cst_vs_multem_tab} приведены данные по относительной ошибке для двух рассмотренных инструментов.  


\begin{equation*}
	\delta\lambda_0 = \frac{|\lambda_0^{CST}-\lambda_0^{MULTEM}|}{\lambda_0^{MULTEM}} \cdot 100\%
\end{equation*}

\begin{table}[]
	\centering
	\caption{cst vs multem tab}\label{cst_vs_multem_tab}
	\begin{tabular}{|c|c|c|c|}
		\hline
		\multicolumn{2}{|c|}{\textbf{Время расчета, с}} & \multicolumn{2}{c|}{\textbf{Ошибка, \%}} \\ \hline
	MULTEM	& CST &  MULTEM                 & CST \\
	\hline
		\multirow{6}{*}{3} & 140 & \multirow{6}{*}{$<3.2 \cdot 10^{-4}$} & $9.2 \cdot 10^{-2}$ \\ 
		& 246 &                  & $2.7 \cdot 10^{-2}$ \\  
		& 502 &                   &  $8.6 \cdot 10^{-3}$\\
		& 1304 &                   & $2.2 \cdot 10^{-3}$ \\
		& 3675 &                   &  $8.3 \cdot 10^{-4}$\\
		& 7200 &                   & $3.2 \cdot 10^{-4}$ \\ \hline	
	\end{tabular}
	
\end{table}


\section{Выводы по главе}



\chapter{\textbf{Программная реализация режима мультипольного разложения}}
\section{Введение}
Мультипольный подход является мощным инструментом при анализе взаимодействия электромагнитного поля с веществом. Основное преимущество мультипольного подхода состоит в том, что он обеспечивает представление произвольного распределения поля в виде суперпозиции полей, созданных набором одиночных мультиполей. Анализ вкладов различных мультиполей может помочь, например, при решении обратной задачи.

Также мультипольный подход является ключевым при построении многих распространенных численных алгоритмов решения задач рассеяния на объектах произвольной формы, к которым относят метод Т-матриц, который рассмотрен в разделе \ref{sec:T_matrix} и метод матриц рассеяния (ссылки из статьи фитцпатрика 28, 27).
   
Однако в оригинальной версии MULTEM отсутствовала возможность применять мультипольное разложение, MULTEM в расчете учитывал мультиполи всех порядков и ориентаций вплоть до заданного значения $l_{max}$. В данной главе будет описана математическая модель одномультипольного приближения и ее программная реализация на языках программирования Fortran и Python с применением конфигурационного ini-файла. 

\section{Математическая модель мультипольного разложения}
Для сферы в силу симметрии Т-матрица является диагональной и вид выражений (\eqref{Tmatrix1}--\eqref{Tmatrix3}) упрощается. Так как MULTEM работает только со сферическими рассеятелями, то в его работе используются только две диагональные матрицы вида:

\begin{equation*}
	T_{ee(hh)} = \begin{pmatrix}
		T_{ee(hh)}^{11} & 0 & ... 	& 0\\
		0 & T_{ee(hh)}^{22} & ... 	& 0\\
		...			&	...	& ... 	& 0\\
		0			&	0	&  0 	& T_{ee(hh)}^{l_{max}l_{max}}
	\end{pmatrix}
\end{equation*}	

При расчетах в MULTEM Т-матрицы используются только в одном месте при решении системы линейных алгебраических уравнений вида:

\begin{equation*}
	 \begin{pmatrix}
		I - T_{ee}\Omega_{ee} 	& 	T_{ee}\Omega_{eh} \\
		T_{hh}\Omega_{he}		&	I - T_{hh}\Omega_{hh} \\
	\end{pmatrix} \cdot 
\begin{pmatrix}
	b_{+e} \\
	b_{+h}	\
\end{pmatrix} = 
\begin{pmatrix}
		T_{ee}a_{0e} \\
		T_{hh}a_{0h}	\\
\end{pmatrix}
\end{equation*}	

где то, то и то

Таким образом, путем искуственного зануления коэффициентов Т-матрицы, соответсвующих определенному порядку мультиполя, можно исключить его влияние на общую картину рассеяния. Например, при $T_{ee+}^{11} = 0$ полностью исключается вклад электрического диполя. 

Для того, чтобы управлять не только порядками мультиполей, но и их ориентацией необходимо занулить все коэффициенты рассеянного поля, не соответствующие заданным числам $l$ и $m$, которые определяет пользователь перед запуском расчета.
	

\section{Реализация режима мультипольного разложения}
\begin{comment}
Режим мультипольного разложения можно реализовать путем модернизации элементов Т-матрицы.  рассматривалась теория Ми и  понятие Т-матрицы. 
Лучше в нотации Борена Хафмана

$\begin{pmatrix}	f_{mn}  \\
					g_{mn}   \end{pmatrix} = 
T\begin{pmatrix}	a_{mn}  \\
					b_{mn}   \end{pmatrix}$

Более подробный вид матрицы


Для реализации режима мультипольного разложения необходимо оставить в Т-матрицы только те элементы, которые отвечают определенным порядкам ВСГ, при этом все остальные элементы обнуляются. В таком случае в картине рассеяния будут наблюдаться только заданные пользователем мультипольные отклики.

Т-матрицы в программном коде MULTEM используются в субрутинах PCSLAB и SETUP (ссылка на 2 главу).

Про конкретный алгоритм, зайти в код - посмотреть что зануля
возможно добавить программную реализацию
\end{comment}
Описать программную реализацию, структуру конфиг файлов

\section{Верификация результатов}

После программной реализации режима мультипольного разложения была проведена верификация на основе данных из работы (ссылка). В указанной работе был произведен мультипольный анализ двумерной периодической структуры, состоящей из золотых сфер, упакованных в гексагональную решетку. В работе были рассмотрены влияния различных мультипольных вкладов, которые по-разному ведут себя в зависимости от угла падения ЭМ волны с фиксированной частотой. В таблице \ref{multipoles} приведена расшифровка мультиполей, которые приведены на рисунке \ref{fig:mul_exp_verification}.

\begin{table}[h]
	\centering
	\caption{multipoles - ссылка на Рокштуля}\label{multipoles}
	\begin{tabular}{|c c c c|}
		\hline
		& $\mathbf{l}$ & $\mathbf{m}$     & \textbf{тип}           \\ \hline
		$p_s$ & 1 & -1,1  & электрический \\
		$m_z$ & 1 & 0     & магнитный     \\ 
		$q_{ks}$ & 2 & -2, 2 & электрический \\ \hline
	\end{tabular}
\end{table} 

\begin{figure}[h!]
	\begin{center}
		\includegraphics[width=0.9\textwidth]{images/verification_design.jpg}
		\caption{verification design}
		\label{fig:verification_design}
	\end{center}
\end{figure}


\begin{figure}[h!]
	\begin{center}
		\includegraphics[width=0.9\textwidth]{images/mul_exp_verification.jpg}
		\caption{mul exp ver}
		\label{fig:mul_exp_verification}
	\end{center}
\end{figure}

Из приведенного сравнения видно, что результаты совпадают, что позволяет сделать вывод о достоверности результатов, которые получены при помощи MULTEM в режиме мультипольного разложения.

\section{Выводы по главе}
\chapter{\textbf{Исследование эффекта углового закрепления связанных состояний в континууме в двумерных периодических структурах}}

\section{Введение}
Усиление ЭМ волн является одной из основных задач, решаемых в оптике, радиофизике, фотонике, плазмонике и других схожих областях. Одним из способов усиления ЭМ поля является создание структур, позволяющих локализовать свет в интересующей области. Для этих задач хорошо подходят высокодобротные системы, выполняющие роль резонатора. Есть несколько спобов достичь высокой добротности резонатора:
\begin{itemize}
	\item за счет увеличения размера резонатора;
	\item за счет использования фотонного кристалла, которые можно использовать в качестве нанорезонатора, или планарного фотонного кристалла, который будет выступать мембраной в резонаторе (ссылка на работу из иоффе);
	\item за счет использования эффекта полного внутреннего отражения в резонаторах с модами шепчущей галереи(3-5 из одита выбрать);
	\item использование относительно новых физических явлений, которые могут значительно увеличить добротность системы, к которым относятся ССК.  
\end{itemize}

Сами по себе ССК являются математическим объектом, который обладает бесконечной добротностью и, соответственно, крайне узким резонансом. К тому же, ССК можно наблюдать только в идеальных бесконечных структурах без потерь [ссылки 22-24 из статьи С.Лепешова]. На практике в реальных структурах рассматриваются квази-ССК, которые обладают конечной добротностью.

Так, например, локализация света в субволновых фотонных структурах возможна на основе диэлектрических частиц с высоким показателем преломления (6-8 одит), в которых будут наблюдаться ССК. Также на сегодняшний день ССК были обнаружены в фотонных кристаллах(16,17), массивах волноводов(18, статья просидинг) и метаповерхностях(19). Данный физический феномен находит различные применения при создании лазеров(20,21), биосенсоров(22), нелинейных преобразователей частоты(10) и других оптических приборов, принцип работы которых основан на локализации света \cite{2021_odit_supercavity_modes}.

\section{Связанные состояния в континууме (все ссылки не надо, только самые знакомые работы)}

ССК впервые было теоретически предсказаны в 1929 году в работе по квантовой механике Ю.Вигнера и Д.фон Неймана~\cite{von1993merkwurdige}, ими было обнаружено, что помимо связанных состояний, которые обладают дискретным энергетическим спектром, и свободных состояний с непрерывным спектром существуют также связанные состояния, энергетический спектр которых является непрерывным. Такие состояния не могут излучать энергию в свободное пространство, за счет чего добротность таких состояний является бесконечной. В силу волновой природы, данное явление наблюдается не только в квантовой механике, но также и в других научных областях, включая фотонику.

\begin{figure}[h]
	\begin{center}
		\includegraphics[width=0.9\textwidth]{images/bic_concept.png}
		\caption{bic concept}
		\label{fig:bic_concept}
	\end{center}
\end{figure}

В фотонике ССК были экспериментально обнаружены в 2013 году~\cite{2013_hsu_nature} (дать описание)  ()адаптированная илюстрация).

Таблица классификации по механизмам

Ниже будет приведено более подробное описание типов ССК, представленных в таблице. 

\subsection{Защищенные симметрией ССК}
Когда система симметрична относительно отражения или поворота, тогда моды, характерные различным классам симметрии, являются полностью развязанными. В таком случае достаточно просто найти связанное состояние одного класса симметрии в непрерывном спектре (континууме) другого. Такие моды остаются развязанными до тех пор, пока сохраняется симметрия.

Защищенные симметрией ССК достаточно просто продемонстрировать на примере расппространения звука в акустическом волноводе и плоской пластины, помещенной посередине (рисунок). Давление воздуха $P$ удовлетворяет скалярному уравнению Гельмгольца с граничными условиями Неймана на поверхности стенок волновода и пластины. Волновод поддерживает распространение волн в направлении $x$. При этом в направлении $y$ моды волновода разделяются на четные и нечетные относительно центра волновода. Нечетная мода имеет критическую частоту $\frac{\pi c_s}{h}$, где  $c_s$ -- скорость звука, а $h$ -- ширина волновода. Пластина заркально симметрична и, соответственно, моды, локализованные вблизи пластины, тоже разделяются на четные и нечетные.

доописать про акустический волновод

Похожие защищенные симметрией ССК наблюдаются в поверхностных волнах на воде~\cite{2013_dyn_loc_phen}, кватовых нитях~\cite{2009_bic_in_quantum_systems}. Защищенные симметрией ССК наблюдаются в $\Gamma$-точке двумерных периодических структур, например, в планарных фотонных кристаллах~\cite{2014_top_nature_bic, 2019_zarina_multipolar_bic}.


\subsubsection{Разделяемые ССК}
ССК могут возникать при решении задачи на собственные числа методом разделения переменных. Гамильтониан двумерной системы представляется в виде $H=H_x(x)+H_y(y)$, где $H_x$ воздействует только на переменную $x$, а $H_y$ -- на $y$. В ходе решения одномерных задач на собственные числа $H_x\psi_x^{(n)}(x)=E_x^{(n)}\psi_x^{(n)}(x)$ и $H_y\psi_y^{(m)}(y)=E_y^{(m)}\psi_y^{(m)}(y)$
волновая функция представляется в виде $\psi = \psi_x^{(n)}(x)\psi_y^{(m)}(y)$, где оба множителя соответствуют связанным состояниям, при этом их суммарная энергия лежит в непрерывном спектре. Такой тип ССК был впервые предложен Робником (66 ссылка) и изучался в различных квантовых системах(67-69) и решении уравнений Максвелла для двумерных структур. Данный вид ССК пока предсказан только теоретически, но существуют гипотезы о системах, в которых может быть получено экспериментальное подтверждение данному типу ССК. К таким системам относятся оптические ловушки для холодных атомов и решетки, которые могут быть описаны приближением сильно связанных электронов (74).

\subsubsection{ССК, возникающие при изменении параметров}
При небольшом числе открытых дифракционных каналов настройки параметров системы может быть достаточно для того, чтобы полностью подавить излучение. Если излучение характеризуется  $n$ степенями свободы, как минимум $n$ параметров должны быть настроены для обнаружения ССК. Подавление излучения можно рассматривать со стороны деструктивной интерференции нескольких изулающих компонент. Здесь возможны три различных сценария, описанных ниже.

\subsubsection{Фабри-Перо ССК} 
Данная система похожа на резонатор Фабри-Перо, в котором в качестве стенок выступают две резонансные структуры. Резонасные структуры подбираются таким образом, чтобы компенсировать любое излучение вне данной системы и локализовать свет внутри. ССК образуются, когда расстояние между структурами $d$ подбирается таким образом, чтобы набег фазы, который приобретает свет при прохождении от одной стенки к другой был кратен $2\pi$. 

Фабри-Перо ССК обычно наблюдаются в системах с двумя идентичными резонансами, настроенными на частоту излучения, например, пара квантовых точек, соединенных квантовой нитью (89-93 посмотреть публикации), две металлические цепочки на металлической подложке (94 посмотреть публикации), в фотонике 96-98, 99, 100, 101, 102, в акустических волноводах (103). Уникальной особенностью данных ССК является то, что несколько резонаторов могут иметь сильное взаимодействие через излучение даже при достаточном отдалении друг от друга. 

\subsubsection{Параметрические ССК на одиночном резонансе} 
Такие виды ССК в англоязычной терминологии еще называются случайными. Возникают тогда, когда одну моду можно представить в виде суперпозиции других мод (19 вики), каждая из которых меняется в зависимости от параметров структуры. В какой-то момент происходит деструктивная интерференция и наблюдается ССК данного типа.

Такие ССК в основном рассматриваются в планарных фотонных кристаллах[41 вики]. Помимо фотонных периодических структур[43][44], ССК встречаются для волн на воде[45], квантовых волноводах[46], механических резонаторах[47], и в виде поверхностных волн[48]. (все вики) Такие ССК наблюдаются в линейных периодических структурах прямоугольников (134, 135), цилиндров (136) и сфер (137).

Данные ССК не гарантируются симметрией, они являются устойчивыми к небольшим изменениям параметров системы. Также такие ССК характерны не только для периодических структур, что было теоретически показано в работах по акустическим волноводам, ссылка какая-то из 143-148), квантовым волноводам с примесями (149-151) и механическим резонаторам (154).

\subsubsection{ССК Фридриха-Винтгена} 
В Фабри-Перо ССК две резонансные структуры разнесены друг от друга в отличие от ССК Фридриха-Винтгена, которые возникают в одной резонасной структуре за счет деструктивной интерференции, которая может наблюдаться при излучении разных мод в один и тот же радиационный канал. В общем случае возникновение таких ССК возможно, когда число резонансов превосходит число радиационных каналов, но при этом растет и число "подстроечных" параметров~\cite{2003_discrete_charging_quantum_dots}.

ССК Фридриха — Винтгена рассматривались в атоме водорода в магнитном поле[26] и экспериментально проявлялись в подавлении автоионизации атома бария[27], топологических изоляторах с дефектами[28] и многих других квантовых системах[29][30]. Также этот тип ССК встречается в двумерных и трехмерных цилиндрических открытых акустических резонаторах[31][32]. В фотонике случайные ССК в периодических структурах могут иногда появляться как ССК Фридриха-Винтгена[33][34]. Примечательным примером является теоретическая реализация ССК на слоистой сферической наночастице из специфически подобранных материалов[36][37], при этом ССК появляются за счет взаимодействия различных дипольных мод одной и той же частицы. По тому же механизму могут быть реализованы высокодобротные состояния и в несферических диэлектрических резонаторах[38][39], но в этом случае моды состоят из бесконечного числа мультиполей[40], а добротность не уходит полностью в бесконечность.





\begin{comment}
\subsection{Мультипольный анализ ССК}
Для обнаружения и анализа ССК в современной науке используется множество методов:
mode expansion method[34 Zarina]
resonance state expansion method[35,36]
итд по Зарине

В последние несколько лет в нанофотонике электромагнитная мультипольная теория (уточнить термин) является доминирующим подходом при анализе резонансов системы (в таком виде совсем не нравится предложение)
Основным преимуществом метода мультипольного разложения является представление произвольного распределения поля в виде суперпозиции полей, образованных определенным набором мультиполей [60,61 Zarina]. Такой метод широко применяется при определении поляризации и диаграмме направленности рассеянного излучения от плазмонных и диэлектрических структур. Это может применяться для поляризаторов[61], диэлектрических наноантеннах[65] и где то еще.

С помощью ММР были объяснены некоторые недавно обнаруженные оптические эффекты - анапольное состояние и эффект Керкера (ссылки).

Далее по статье Зарины как можно использовать ММР для описания обоих типов ССК.
\end{comment}

\section{Численное исследование углового закрепления ССК}

Численное исследование эффекта углового закрепления осуществлялось в одномультипольном приближении в программе MULTEM. Одномультипольное приближение заключается в том, что в качестве двумерной периодической структуры выступает не привычная решетка, состоящая из реальной частицы определенной формы (сферы в случае MULTEM), а решетка, в которой в каждой элементарной ячейке размещен точечный мультиполь с определенным порядком ($l$) и ориентацией ($m$) (рис. 
\ref{fig:multipolar_lattice_with_spheres}).

\subsection{Мультипольная решетка}
В программе MULTEM данная система является все тем же массивом сфер, упакованных в квадратную решетку, обладающих только одним мультипольным откликом. Важным также является то, что мультиполи, из которых состоит решетка, должны быть одиноковой ориентации, при соблюдении указанных требований такую решетку можно назвать мультипольной. В данной работе рассматривается мультипольная решетка, состоящая из магнитных октополей, ориентированных вдоль оси $z$ (рис \ref{fig:bic_pinning1}а), в терминах квантовых чисел это магнитный мультиполь с $l = 3$ и $m = 0$. 

\begin{figure}[h]
	\begin{center}
		\includegraphics[width=0.9\textwidth]{images/multipolar_lattice_with_spheres.jpg}
		\caption{multipolar lattice with spheres}
		\label{fig:multipolar_lattice_with_spheres}
	\end{center}
\end{figure}

Можно предположить, что мультипольная решетка будет излучать также как и одиночный мультиполь. То есть взаимодействие мультиполей в решетке будет влиять на перераспределение интенсивности излучения, но не изменит диаграмму направленности (дать ссылки из статьи).

\subsection{Угловое закрепление незащищенных симметрией ССК}
При рассмотрении диаграммы направленности магнитного октополя видно, что он имеет ярковыраженный минимум в пределах $\theta \in [0,\pi/2]$. Угол, при котором наблюдается минимум обозначим $\theta_0$, тогда $\theta_0$ можно вычислить аналитически при нахождении экстремума выражения которое определяет диаграмму направленности магнитного октополя (см. приложение ссылка на прилолжение). Решение уравнения $P'_3(\cos\theta)=0$ дает значение $\theta_0 = 63.4^\circ$, что соответствует положению незащищенного симметрией ССК. На рисунке \ref{fig:bic_pinning1}г схематично продемонстрирован процесс образования незащищенных симметрией ССК в мультипольных решетках. Синяя линия является дисперсионной кривой мультипольной решетки и определяется резонасным взаимодействием решетки с падающей на нее волной. Красная пунктирная линия характеризует нодальную линию мультиполя -- линию, вдоль которой отсутствует излучение. На пересечении этих двух кривых образуется незащищенное симметрией ССК, что и продемонстировано на карте коэффициента отражения мультипольной решетки (рис. \ref{fig:bic_pinning1}в).

 
\begin{figure}[h]
	\begin{center}
		\includegraphics[width=0.9\textwidth]{images/bic_pinning1.jpg}
		\caption{bic pinning 1}
		\label{fig:bic_pinning1}
	\end{center}
\end{figure}
 
Крайне важным здесь является то, что дисперсионная кривая зависит от параметров системы, в частности периода решетки, в отличие от нодальной линии. В связи с этим для любой мультипольной решетки можно аналитически определить положение незащищенного симметрией ССК, которое к тому же будет сохранять свое положение при изменении периода решетки. На рисунке \ref{fig:bic_pinning2} представлены семейства карт коэффициента отражения при разных отношения периода решетки к диаметру частицы. Видно, что положение незащищенного симметрией ССК остается неизменным. 

Так как рассматриваемая система является субволновой, то она имеет только один дифракционный канал. При увеличении периода решетки дисперсионная кривая поднимается вверх по частоте и видно, что при достижении порога дифракции незащищенный симметрией ССК практически исчезает (рис. \ref{fig:bic_pinning2}г), так как открываются дополнительные дифракционные каналы, через которые может происходить излучение. 

\begin{figure}[h]
	\begin{center}
		\includegraphics[width=0.9\textwidth]{images/bic_pinning2.jpg}
		\caption{bic pinning 2}
		\label{fig:bic_pinning2}
	\end{center}
\end{figure}


\section{Выводы по главе}


\backmatter %% Здесь заканчивается нумерованная часть документа и начинаются ссылки и
            %% заключение

\chapter{\textbf{ЗАКЛЮЧЕНИЕ}}
Основные результаты работы:
\begin{itemize}
    \item 1 результат;
    \item 2 результат;
    \item 3 результат.
\end{itemize}



\include{05_biblio}


%\subsubsection*{Acknowledgements}

\end{document}

%%% Local Variables:
%%% mode: latex
%%% TeX-master: t
%%% End:
 