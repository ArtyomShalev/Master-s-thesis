\chapter{\textbf{Введение}} 

С начала 21 века огромной популярностью пользовались объемные искусственные материалы - метаматериалы. Такие материалы позволяют реализовать электромагнитные свойства вещества, которые отсутствуют в природе. Однако из-за высоких потерь в материале, высокой дисперсии и сложности в изготовлении в микро- и наномасштабах было решено перейти в планарным метаматериалам - метаповерхностям. Метаповехрности относительно просты в изготовлении за счет современных методов литографии и 3D-печати, а субволновая толщина метаповерхности значительно снижает потери~\cite{2016_hsu_bic_review}.

Также с 2013 года экспериментально были обнаружены связанные состояния в континууме в двумерных периодических структурах, в частности в планарных фотонных кристаллах ~\cite{2013_hsu_nature}, которые открыли новое научное направление в фотонике, связанное с объяснением физики ССК и их применений. 

График по количеству публикаций и примеры использования.
(также график и знаковые работы)

Современное качественное научное исследованием невозможно представить без проведения численного моделирования. В связи с этим возникает острая необходимость проводить численный расчет подобных двумерных периоидических структур. Как правило для таких задач используют прямые численные методы, которые лежат в основе работы современных пакетов электромагнитного (ЭМ) моделирования. Такой подход остается самым распространенным в силу своей универсальности, однако он не лишен недостатков. В качестве альтернативы предлагается использовать полуаналитические методы, например метод РГУ, и еще методы решения подобных задач. Основным преимуществом полуаналитических методов, конечно же, является скорость расчета. 

Одним из способов проектирования новых метаповерхностей является решением обратной задачи. При таком методе по заданному физическому отклику системы (фронт фаза интенсивность расширить) подбирается оптимальная конфигурация как отдельных рассеивателей, так и всей системы в целом. Одним из эффективных современных подходов к решению обратных задач является применение различных алгоритмов машинного обучения. Для обучения моделей необходимы (многомиллионные или сколько там) выборки, в связи с чем к скорости решения прямых задач и, соответсвенно, к выбору инструментов нужно подходить очень внимательно.

Краткий обзор глав, что будет сказано. 


Тем не менее возможность моделирования двумерной периодической структуры с определенным мультипольным откликом может приводить к обнаружению новых эффектов, что будет описано в главе, посвященной




