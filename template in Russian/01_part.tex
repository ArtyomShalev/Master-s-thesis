\chapter{\textbf{Численные методы решения задач рассеяния света на двумерных периодических структурах}}
\label{1chapter}

\section{Численные методы решения уравнений Максвелла}
\label{sec:num_Maxwell}
Для решения любой ЭМ задачи можно воспользоваться прямыми методами решения уравнений Максвелла. Достоинством такого подхода является универсальность, однако эти методы практически всегда будут медленными. Существует множество разновидностей численных методов, которые используются в современных программных пакетах ЭМ моделирования. Однако чаще всего они будут являться производными от методов, описанных ниже: 

\begin{itemize}
	\item методы решения ур.Максвелла в дифференциальной форме, основанные на конечных разностях. Конечные разности составляются для временной (\textbf{FDTD} -- Finite Difference Time Domain) и частотных (\textbf{FDFD}~--~Finite Difference Frequency Domain) областей;
	\item метод конечных элементов (\textbf{FEM}~--~Finite Elements Method);
	\item метод граничного элемента, который зачастую упоминается как метод моментов (\textbf{MoM}~--~Method of Moments). 
\end{itemize}

FDTD является одним из самых распространенных методов решения электродинамических задач. Данный метод требует несколько операций (6 в трехмерном случае -- для каждой компоненты электрического и магнитного полей) на один узел сетки пространственно-временной сетки, поэтому этот метод крайне неэффективен в случае больших расчетных областях. Также недостатком данного метода является то, что он работает только со структурированными сетками (рис \ref{fig:test}).

FEM лишен данного недостатка и разбивает расчетную область на неструктурированную сетку (треугольную для двумерных задач и тетраедральную для трехмерных) (рис \ref{fig:test}), но данный метод значительно медленнее при расчетах во временной области. FEM обычно используется для вычисления гармонических полей и является стандартным методом для расчета вихревых токов.

\begin{figure}[h]
	\centering
	\begin{subfigure}{.5\textwidth}
		\centering
		\includegraphics[width=.7\linewidth, height=5 cm]{images/regular_grid.jpg}
		\caption{структурированная сетка}
		\label{fig:sub1}
	\end{subfigure}%
	\begin{subfigure}{.5\textwidth}
		\centering
		\includegraphics[width=.7\linewidth, height=5 cm]{images/irregular_grid.jpg}
		\caption{неструктурированная сетка}
		\label{fig:sub2}
	\end{subfigure}
	\caption{Виды сеток в различных численных методах}
	\label{fig:test}
\end{figure}

MoM работает с уравнениями Максвелла в интегральной форме, в качестве неизвестных выступают сторонние источники. Такой метод лучше подходит для открытых расчетных областей и небольших токопроводящих поверхностей.

Альтернативным подходом является использование полуаналитических методов. Например, для решения задачи рассеяния ЭМ волны на сферической частице используется теория Ми, которая позволяет получить точное аналитическое решение. 


\section{Основные сведения из теории Ми}
\label{sec:Mie_theory} 

Теория Ми рассматривает рассеяние ЭМ волны сферической частицей, такое рассеяние также называют рассеянием Ми. Данная задача является классической задачей электродинамики, которая была решена аналитически Густавом Ми в 1908 году~\cite{Mie}. Теория Ми является общей и приводит к различным частным случаям при рассмотрении  определенного отношения диаметра частицы к длине волны:

\begin{itemize}
	\item в случае, когда диаметр частицы много меньше длины волны $d << \lambda$, происходит Рэлеевское рассеяние. При данном рассеянии длина волны света не изменяется, такое рассеяние также называется упругим. Внешнее ЭМ поле поляризует частицу, возбуждая в ней переменный дипольный момент. Такой дипольный момент переизлучает свет с частотой колебания диполя и дипольной диаграммой направленности;
	\item в случае, когда $d \approx \lambda$ или чуть меньше (частица становится субволновой), диаграмма направленности становится более сложной в силу возникновения интерференции волн, отраженных от различных участков поверхности сферы. Такой случай называется резонансным, так как в рассеивающей сфере может укладываться несколько длин волн. При этом возможны ситуации, когда в рассеянии значительно доминирует вклад какой-то определенной векторной сферической гармоники (ВСГ), математический смысл которой будет рассмотрен далее. Тогда на больших расстояниях от частицы диаграмма направленности рассеянного поля будет похожа на соответствующую диаграмму направленности угловой части ВСГ. 
	\item в случае, когда $d >> \lambda$, поверхность сферы будет в первом приближении вести себя как плоская поверхность, и рассеяние света можно в терминах отражения и преломления в соответствии с формулами Френеля.  
\end{itemize}  

Математика в теории Ми относительно проста, однако местами громоздка. Представить разложение ЭМ полей во всем пространстве (особенно с высокой точностью) в виде рядов  до появления вычислительной техники было поистине трудоемкой задачей. Однако с развитием электронно-вычислительной техники строгое решение задачи Ми стало значительно проще. Преимуществами использования теории Ми являются сокращенный объем вычислений, что значительно увеличивает скорость расчета, и возможность оценивать вклады различных мультиполей в общую картину рассеяния. Ниже будет кратко рассмотрен ход решения задачи рассеяния плоской волны сферической частицей. 


Для решения векторных волновых уравнений удобно перейти от векторной функции к скалярной. В сферической системе координат можно ввести вспомогательные векторные функции которые будут связаны со скалярной функцией следующим образом:

\begin{equation*}
\mathbf{M} = \nabla \times (\mathbf{c}\psi)
\end{equation*}

\begin{equation*}
\mathbf{N} = \frac{\nabla \times \mathbf{M}}{k} 
\end{equation*}

где $\psi$ -- скалярная функция, которая является решением скалярного волнового уравнения, $\mathbf{c}$ -- постоянный вектор, а $k$ -- волновое число.

Данные функции обладают всеми необходимыми свойствами ЭМ поля -- они удовляетворяют векторному волновому уравнению, их дивергенции равны нулю, а также ротор $\mathbf{M}$ пропорционален $\mathbf{N}$ и наоборот.
Таким образом, задача нахождения решений волнового уравнения для векторного поля сводится к задаче решения скалярного волнового уравнения. Функции $\mathbf{M}$ и $\mathbf{N}$ называются магнитными и электрическими ВСГ соответственно.

В ходе решения скалярного волнового уравнения, который подробно описан в книге~\cite{Bohren1998} получается, что функция $\psi$ разделяется на четную и нечетную и имеет вид:
\begin{equation} \label{eq:psi1}
	\psi_{emn} = \mathrm{cos}m\varphi P_n^m(\mathrm{cos}\theta)z_n(kr) 
\end{equation}
\begin{equation} \label{eq:psi2}
	\psi_{omn} = \mathrm{sin}m\varphi P_n^m(\mathrm{cos}\theta)z_n(kr) 
\end{equation}

где $z_n$ -- любая из четырех сферических функций Бесселя.

В таком случае после применений соотношений (\eqref{eq:psi1} -- \eqref{eq:psi2}) ВСГ также разделяются на четные и нечетные $\mathbf{M}_{o(e)mn}$ и $\mathbf{N}_{o(e)mn}$. В силу своей громоздкости явный вид ВСГ приведен в приложении (ссылка на него). Стоит отметить, что ВСГ являются функциями волнового числа $k$ и радиус-вектора $\mathbf{r}$, а также то, что верхний индекс ВСГ указывает на то, какая сферическая функция Бесселя будет использована при расчете ВСГ.

После этого можно разложить любое поле по ВСГ. В модельных задачах падающее поле обычно представляется в виде плоской волны, и ее разложение по ВСГ имеет вид:
\begin{equation*}
E_{inc} = E_0e^{ikrcos\theta}\mathbf{e}_x = E_0\sum_{n=1}^{\infty} i^n \frac{2n+1}{n(n+1)}(\mathbf{M}_{o1n}^{(1)}-i\mathbf{N}_{e1n}^{(1)})
\end{equation*}

\begin{equation*}
H_{inc} = \frac{-k}{\omega\mu}E_0\sum_{n=1}^{\infty} i^n \frac{2n+1}{n(n+1)}(\mathbf{M}_{e1n}^{(1)}+i\mathbf{N}_{o1n}^{(1)})
\end{equation*}

Рассеянные поля имеют вид:
\begin{equation*}
 	E_{s}= \sum_{n=1}^{\infty} E_n (ia_n\mathbf{N}_{e1n}^{(3)} - b_n\mathbf{M}_{o1n}^{(3)}) 
\end{equation*}

\begin{equation*}
	H_{s} = \frac{k}{\omega\mu}\sum_{n=1}^{\infty} E_n (a_n\mathbf{M}_{e1n}^{(3)} + ib_n\mathbf{N}_{o1n}^{(3)}) 
\end{equation*}

где $E_n = \frac{iE_0(2n+1)}{n(n+1)}$, а верхний индекс $(3)$ у ВСГ показывает то, что радиальная зависимость функций (\eqref{eq:psi1} -- \eqref{eq:psi2}) используются сферические функции Ханкеля.

Поля внутри сферической частицы имеют вид:
\begin{equation*}
	E_{i}= \sum_{n=1}^{\infty} E_n (-id_n\mathbf{N}_{e1n}^{(1)} + c_n\mathbf{M}_{o1n}^{(1)}) 
\end{equation*}

\begin{equation*}
	H_{i} = \frac{-k_1}{\omega\mu_1}\sum_{n=1}^{\infty} E_n (d_n\mathbf{M}_{e1n}^{(1)} + ic_n\mathbf{N}_{o1n}^{(1)}) 
\end{equation*}

После применения граничных условий получаются выражения для частотной зависимости коэффициентов $c_n(\omega)$, $d_n(\omega)$, $b_n(\omega)$ и $a_n(\omega)$.
В приложении (ссылка на него) приведены полные выражения для коэффициентов разложения полей, которые получаются после применения граничных условий.

\section{Дополнение теории Ми на случай несферических частиц. Метод Т-матриц}
\label{sec:T_matrix}
Опираясь на основные положения теории Ми, можно перейти к рассмотрению метода Т-матриц, который является дополнением теории Ми на случай несферических частиц, которые повсеместно встречаются как в природе, так и в искуственных рассеивателях с произвольной конфигурацией, придуманной человеком. Чисто аналитическое решение задач рассеяния ЭМ поля на частицах такой формы невозможно, в связи с чем возникает потребность в создании высокоэффективных численных методов решения таких задач. 

В формализме Т-матриц ЭМ волны также как и в теории Ми раскладываются на ВСГ. Вся информация о мультипольном отклике частицы содержится в ее Т-матрице. Т-матрица связывает коэффициенты разложения электрического и магнитного рассеяннного и падающего полей следующим образом:

\begin{equation} \label{Tmatrix1}
	\begin{pmatrix}	b_{e}  \\
		b_{h}   \end{pmatrix} = 
	T\begin{pmatrix}	a_{e}  \\
		a_{h}   \end{pmatrix}
\end{equation}

В общем случае в Т-матрице также содержатся члены, описывающие влияние падающего электрического поля на рассеянное магнитное и наоборот:
\begin{equation} \label{Tmatrix2}
		Т = 
		\begin{pmatrix}	
			T_{ee} & T_{eh}  \\
			T_{he} & T_{hh}  
		\end{pmatrix}
\end{equation}

А матрица $T_{ee}$ имеет вид:
\begin{equation} \label{Tmatrix3}
	T_{ee} = \begin{pmatrix}
		T_{ee}^{11} & T_{ee}^{12} & ... \\
		T_{ee}^{21} & T_{ee}^{22} & ... \\
		...			&	...		  & ... \\
	\end{pmatrix}
\end{equation}	


Т-матрица не зависит от типа падающего поля и ориентации частицы в пространстве. Т-матрица зависит только от размера частицы, ее материальных параметров, формы и длины волны света. Можно сказать, что Т-матрица содержит полную информацию о способности частицы рассеивать падающее на нее ЭМ поле разной частоты. 

Это приводит как минимум к двум весомым преимуществам при использовании формализма Т-матриц при решении задач рассения:
\begin{itemize}
	\item если Т-матриц расчитана для одной ориентации частицы, то для другой ориентации (или направления падающего поля) ее достаточно пересчитать аналитически;
	\item Т-матрицы используются при моделировании систем, состоящих из множества случайно ориентированных частиц, так как с их помощью можно получить выражения в замкнутой форме.  
\end{itemize}


Большую часть расчетного времени в данном методе занимает расчет большого числа поверхностных интегралов, которые требуют расчетов функций Бесселя и ВСГ. Стоит отметить, что любые численные методы решения ЭМ задач, которые лежат в основе современных программных пакетов, способны посчитать Т-матрицу. Однако методы, рассмотренные ниже дают полуаналитическое решение.

Аналитические выражения для Т-матриц для сфер можно получить из теории Ми. Однако для расчета Т-матриц для несферических частиц существует ряд методов, к которым относят метод расширенных граничных условий (\textbf{EBCM} -- extended boundary condition method) и дискретное дипольное приближение (\textbf{DDA} -- discrete dipole approximation), которые подробно рассмотрены в работах~\cite{2015Kahnert}~\cite{2014TMatrixComparison} 

\begin{comment}
\subsection{DDA} 
Вообще надо смотреть 25 ссылку и как там этот метод используется для вычисления Т-матриц

По сравнению с FDTD метод DDA решает уравнения Максвелла в частотной области. Рассмотрение данного метода начинается с уравнений:

\begin{equation*}
	\nabla \times \mathbf{H} + i\omega\varepsilon\mathbf{E} = 0
\end{equation*}

\begin{equation*}
	\nabla \times \mathbf{E} - i\omega\mu\mathbf{H} = 0
\end{equation*}

формулы выше скорее всего не нужны

Рассмотрим лучай диэлектрической частицы в неоднородной среде:
\begin{equation*}
	\nabla \time \nabla \mathbb{E(r)} - k_0^2\mathbf{E(r)} = i\omega\mathbf{j(r)}
\end{equation*}

итд расшифровки

Решение - функци Грина

В итоге решение волнового уравнения (ссылка)


пояснение к формуле

Уравнение (ссылка) также называется объемным интегральным уравнением. Данное уравнение можно решать разными способами (МОМ может быть), но в данном случае будет рассмотрен DDA, который был впервые предложен в работе (14 ссылка из ревью).

Основная идея ДДА заключается в разбиении разбить объем частицы на множество маленьких объемов. Это можно сделать по-разному, но самый простой и очевидный способ -- кубическая сетка.
РИСУНОК можно 

Из рисунка также видно, что по сравнению с методом FDTD дискретизации подвергается только сам объем частицы.
После дискретизации уравнение (ссылка) переходит в СЛАУ, которая решается стандартными методами.

можно переписать интеграл в виде суммы но лучше не стоит

Дискретизация должна быть такой, чтобы каждый элементарный объем был много меньше длины волны. В таком случае можно пренебречь любыми неравномерными распределениями амплитуды поля по элементарному объему. С физической точки зрения это значит, что все заряды в элементарном объеме осциллируют в фазе

В таком случае можно представить, что в каждом элементарном объеме колеблются одиночные диполи. Тогда вся структура описывается массивом одиночных диполей, расстояние между которыми определяется шагом сетки.

Если опустить промежуточный вывод

то приходим к СЛАУ, которое связывает какое то (наведенное виимо) поле и падающее поле в виде:
\begin{equation*}
	\mathbf{E_inc^m} = \sum_{k=1}^{M} \mathbf{Q_{mn}} \cdot \mathbf{E_{exc}^k}
\end{equation*}

СЛАУ решается стандартными методами

Чтобы определить поле в произвольной точке вне частицы, то нужно воспользоваться интегральным уравнение (ссылка), в котором аргумент $\mathbf{r}$ будет принадлежать области вне частицы. А в источнике (в какой там части) будет учтено поле наводимое частицей. Плотность тока в каждом элементарном объеме можно выразить через наведенный дипольный момент $\mathbf{j_m} = \frac{\mathbf{p_m}\omega}{iV_m}$, также дипольный момент связан с наведенным частицей полем соотношением $\mathbf{p_m} = \mathbf{t_m} \cdot \mathbf{E_exc^m}$, где $\mathbf{t_m}$ матрица поляризуемости ячейки $m$. Тогда (интегральное уравнение ссылка) можно переписать в виде (61)

Для решения задачи рассеяния остается только подобрать такую модель поляризуемости, которая бы корректно описывала диэлектрические свойства частицы. 

Стоит отметить, что данный алгоритм не учитывает форму и характер распределения материальных параметров частицы. Это значит, что данный метод применим для произвольной формы частицы.    
\end{comment}

EBCM широко распространен и является одним из наиболее эффективных методов для расчета рассеяния ЭМ волн на частицах произвольной формы. Данный метод впервые был предложен Вотерманом ~\cite{1965Waterman} в цикле его работ в 1965-1979 годах. Данный метод был применен при решении множества электромагнитных задач (расчет диаграмм направленности для периодических структур, состоящих из рассеивателей различных форм, акустических и плазмонных задачах рассеяния и при моделировании многослойных структур) ~\cite{2004MischenkoDatabase}.
Несмотря на такую богатую историю использования и обширный спектр применений, численная реализация данного метода в некоторых случаях (например, для выпуклых рассеивателей с высоким аспектным отношением) остается до сих пор актуальной ~\cite{2011LeRu}.


\section{Двумерные решеточные суммы(еще в разработке)}
Изучение распространения волн в периодических структурах приводит к ряду интересных математических задач (каких?)
Особенностью решения подобных задач является то, что решение на бесконечной структуре можно свести к решению задачи в одной элементарной ячейке. Существует множество подходов решения подобных задач, одним из которых является использование функций Грина. Однако, если функции Грина для бесконечной области достаточна проста, то использование подхода функций Грина для одной элементарной ячейки приводит к появлению решеточных сумм вида $\sum_\Lambda \mathrm{exp}(i\bm{\beta} \bm{R_m})G_0(\bm{r},\bm{R_m})$, где $\Lambda$ -- периодический массив, $\bm{R_m}$ -- радиус-вектор, $\bm{\beta}$ -- постоянная распространения или в терминах периодических структур Блоховский вектор, а $G_0(\bm{r},\bm{r'})$ -- функция Грина, которая описывает влияние источника, находящегося в точке $\bm{r'}$ на значение в точке $\bm{r}$ . Произведение $\mathrm{exp}(i\bm{\beta} \bm{R_m})G_0(\bm{r},\bm{R_m})$ обозначают $G_\Lambda$ и называют квази-периодической функцией Грина. Данная функция описывает влияние периодического массива точечных источников с одинаковой амплитудой, но разным сдвигом фаз на поле в точке $\bm{r}$. При решении задач рассеяния ЭМ волн на периодических структурах важно, чтобы $G_\Lambda$ рассчитывалась быстро и точно.

Методы, основанные на вычислении $G_\Lambda$ для уравнения Гельмгольца, нашли себе применения во многих областях физики, например в дифракции медленных электронов (84 линтон), акустике(91 линтон), рассеяние ЭМ волн на экранах с отверстиями(63 линтон), распространение волн в композитных материалах(72 линтон), определении зонной структуры фотонных кристалов(96 линтон) и проектировании конфигурации метаповерхностей(115 линтон).

Есть три основные причины для изучения решеточных сумм:
\begin{itemize}
	\item задачи рассеяния света на периодических структурах часто могут быть сведены к системе линейных уравнений, в которых матрица коэффициентов будет включать решеточные суммы;
	\item в виде коэффициентов разложения ряда решеточные суммы могут быть частью эффективной численной схемы вычисления $G_\Lambda$, так как значения данной функции используется неоднократно;
	\item в начальном виде решеточные суммы представляют из себя члены условно сходящегося ряда. Для численной эффективной реализации вычисления членов такого ряда необходимо провести ряд преобразований, что, в свою очередь, является интересной задачей на стыке математики и компьютерных наук.
\end{itemize}  

Существуют разные подходы для преобразования условно сходящихся рядов в сходящиеся с высокой скоростью. В данной работе внимание будет уделено подходу, основанному на суммировании Эвальда, и численной реализации этого подхода Камбе (ссылки на место в работе)

это удалить скорее всего
С математической точки зрения решеточную сумму можно охарактеризовать как сумму вида $ $, где $\Lambda$ - массив из точек, каждая из которых характеризуеттся радиус-вектором $ $ , а $ $ - заданная функция. (стоит ли говорить про периодичность?). Такие суммы представляют интерес, когда $ $, где $ $ -- функция Грина для определенного уравнения в частных производных, описывающая эффект в точке $ $ от источника в точке $ $, а $ $ - сила источника. Решеточные суммы нашли широкое применение при исследовании электростатических взаимодействий в ионных кристаллах еще в 19 веке (ссылка из Линтона) 

В работе (ссылка на линтона) дается обширный обзор решеточных сумм, которые используются при решении уравнения Гельмгольца. В этом разделе хочется поподробнее остановиться на двумерных решеточных суммах в трехмерном пространстве, ведь именно их вычисление требуется для работы программы MULTEM, которая рассмотрена в главе (ссылка на главу).
  
Для понимания (решеточных сумм) необходимо ввести ряд понятий и ввести их математическое определение:
1) прямая и обратная решетки
2) формула суммирования Пуассона
3) условно сходящиеся ряды (под вопросом)
4) квази-периодическая ф Грина и ее интегральное представление


Вычисление $G_\Lambda$ является фундаментальным при решении задач рассеяния ЭМ волн. Для их вычислений написано множество алгоритмов (дать краткий обзор по ним). Далее будут рассмотрены некоторые представления КПФГ, необходимые для понимания решеточных сумм.
формулки для ФГ точечного источника

Однако данное решение справедливо только для вещественной К, в общем случае К должно быть комплексным. Более того, изучение вихревых решеток в сверхпроводниках требует, чтобы к был чисто мнимым (32 линтон)

Определение КПФГ 

Стоит отметить, что в случае двумерной решетки КПФГ безразмерна, хотя в трехмерном случае имеет размерность обратной длины.


5) двойные ряды

КПФГ представлена в виде суммы по прямой решетке. ФСП позволяет представить данную сумму в виде суммы по обратной решетке. Такие ряды называются двойными 

Аккуратно описать про соответсвие д и длам

Возможно всю эту конкретику во вторую главу, а тут только теорию.

Резюме: 
Для вычисления функции Грина (какой именно функции Грина - расширить ее название) используется реализация Гуерин-Енок-Тайеб (проверить правописание на русском). Альтернативный метод Камбе имеет точность вычисления близкой к машинной и в как минимум в 12 раз быстрее и сходится даже в плоскости решетки.

Альтернативный метод основан на суммировании Эвальда (2,3 мороз) и приводит к экспоненциальной сходимости представления(убрать функции Грина для свободного пространства $G_{0\Lambda}$ и ее решеточных сумм (решеточные суммы функции?). Такое представление было впервые предложено Камбе в 1967 (2,3 мороз)

(нужно ли давать громоздкие формулы?)

С тех пор данное представление является неотъемлемой частью программ для численного расчета, основанных на методе гриновских функций Корринги – Кона – Ростокера, предназначенного для определения зонной структуры периодических структур и теории дифракции медленных электронов, которая описывает дифракцию квантовых и классических волн от двумерных периодических структур в 3Д (ВООООТ??) Эти программы успешно себя показали при решении различных физических задач с 70-ых годов 20 века. Стоит отметить, что данные программные реализации могут успешно решать как скалярные, так и векторные уравнения Гельмгольца.

Реализация Камбе (ур 6-8) не были включены в анализ (1 источник мороз). Во-первых, при известной сигме и к парал идентичные решеточные суммы используются в (ур 4) для расчета $G_{0\Lambda}$ везде, например, в плоскости структуры ($z = 0$) (какое условие?) (для сравнения в источнике 1 ряды сходятся только при $|z|\geq 10^{-6}$)
Во-вторых, не составляет проблемы рассчитать функцию Грина с машинной точностью. В третьих, выражения (6-8) Камбе предоставляют unparallel скорость сходимости по сравнению с (1 ссылка).

После первого вычисления решеточной суммы $D_{lm}$ до заданного количества мультиполей дальнейшие вычисления функции Грина зависят только скорости вычисления функций Бесселя и сферических гармоник.

Таблица сравнения из мороза. Там есть и скорость сходимости суммирования Эвальда-Камбе

\subsection{Суммирование Эвальда} (1995 год суммирование Эвальда)


