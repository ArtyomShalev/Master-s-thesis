\chapter{\textbf{Технический обзор (название поменять)}}

\section{Введение}
MULTEM был впервые представлен в работе ~\cite{1998Stefanou} в 1998 году. Данная программа позволяет рассчитывать зонную структру и коэффициенты прохождения, отражения и поглощения бесконечных двумерных периодических структур. Изначально данная программа решала задачу расчета фотонных кристаллов, однако в ней не предусмотрено ограничений на отношение периода решетки к длине волны излучения. В связи с этим есть предпоссылки для использования данного инструмента при моделировании любых двумерных периодических структур, в то числе и метаповерхностей. MULTEM позволяет моделировать многослойные структуры, при этом слои могут состоять из разного материала. MULTEM может работать только с решетками Браве в каждом слое и только со сферическими рассеивателями. Несмотря на эти ограничения, он по-прежнему актуален, о чем свидетельтствует количество цитирований оригинальной работы за последние 10 лет (привести сколько цитирований по обеим публикациям), а с открытием такого явления в фотонике как связанные состояния в континууме (ССК) (ссылка на работу 2008 года) и их экспериментальным обнаружением в планарном фотонном кристалле ~\cite{2013_hsu_nature}, то для некоторых систем MULTEM может быть актуален даже в своей оригинальной реализации ~\cite{2019_zarina_multipolar_bic}.

\section{Принцип работы MULTEM} 

Программа MULTEM написана в процедурном стиле, поэтому удобно будет рассмотреть математические операции в соответствии с процедурами (или субрутинами на языке программирования Fortran) и функциями, которые их выполняют.

\subsection{Описание входных данных (возможно в приложение вместе с конфиг файлом)}
Чтение входных данных осуществляется через конфигурационный файл, содержимое которого представлено в приложении (ссылка на приложение).

Параметр KTYPE определяет в каком режиме будет производиться расчет. KTYPE принимает значение от 1 до 3. 1 -- направление падающей волны определяется двумя полярными углами $\theta$ и $\varphi$, 2 -- направление падающей волны определяется частотой волны и проекциями волнового вектора $k_x$ и $k_y$, 3 -- программа работает в режиме расчета зонной структуры.

Параметр KSCAN определяет тип сканирования в определенном диапазоне. 1 -- частотное сканирование, 2 -- сканирование по длинам волн.

Параметр KEMB определяет наличие среды вокруг структуры. 0 -- отсутствует, 1 -- присутствует. 

Параметр LMAX определяет максимальный порядок ВСГ, которые будут приняты в рассмотрение. Максимальное значение LMAX = 7.

Параметр NCOMP определяет количество компонентов в одном срезе многослойной структуры, а параметр NUNIT определяет количество таких срезов.

Параметры ALPHA, ALPHAP и FAB определяют геометрию решетки. ALPHA соответствует длине вектора трансляции решетки в одном из направлений и всегда равно 1. ALPHAP соответствует вектору трансляции в другом направлении и задается в количестве ALPHA. FAB определяет угол между двумя векторами трансляций решетки. Таким образом, в случае квадратной решетки ALPHA = ALPHAP = 1, FAB = 90. 

Переменная RMAX?

Параметры ZINF и ZSUP определяют минимальное и максимальное  значение частотного диапазона (или диапазона длин волн в зависимости от выбранного значения KSCAN). Значение частоты и длины волны нормируются на параметр ALPHA. NP определяет количество точек на спектр.

Параметры THETA/AK(1) и PHI/AK(2) определяют полярные углы или проекции волнового вектора на двумерную решетку в зависимости от выбранного значения KTYPE.

Параметр POLAR определяет поляризацию падающей волны и принимает значения S (TE -- электрическое поле параллельно плоскости падения волны) или P (TM -- магнитное поле параллельно плоскости падения волны).

Параметр FEIN определяет направление вектора поляризации (в градусах) и его можно задать только в случае нормального падения волны. 

Параметр IT относится к определению состава элементарного среза структуры. 1 -- однородная плоскость, 2 -- несколько слоев сферических рассеивателей. 

Параметры MUMED и EPSMED определяют комплексные значения материальных параметров среды, в которой моделируется решетка. 

Параметр NPLAN определяет количество плоскостей в одном слое, а NLAYER -- количество слоев.

Параметр S определяет радиус сферы. Данный параметр нормируется на ALPHA.

Параметры EPSSPH и MUSPH определяют комплексные значения материальных параметров сфер.

Параметры EPSEMBL, MUEMBL и EPSEMBR, MUEMBR имеют смысл только в случае моделирования однородной плоскости и определяют материальные параметры верхнего и нижнего полупространства соответственно.


\subsection{Структура программы и основные математические выражения (есть в ворде, нужно перебить в латех -- не успел пока сделать)}
\begin{comment}
Субрутина ПССЛАБ вычисляет матрицы прохождения и отражения для плоскости из сфер в однородной среде (уравнение 46). 
Матрица ЗЕТ (уравнения 27-30) вычисляется по алгоритму, основанному на суммировании Камбэ, которая подобно суммированию Эвальда разбивает сумму по двумерной решетке (ур 27) на две суммы (одна в прямом и другая в обратном пространстве (ур 17)), при этом расчете используется комлексная функция ошибок или функция Фадеевой (о ней будет позже). 

Константы из ур 29 вычисляются в субрутине ЭЛМГЕН и хранятся в массива ЕЛМ. Данные коэффициенты не зависят от решетки ли частоты и поэтому ЕЛМГЕН вызывается один раз в начале. 

ХМАТ за счет свойства 31 перераспределяет элементы матрицы З и переводит ее в блочно-диагональную форму. Матрица З разбивается на две матрицы в зависимости от четности сумм $l+m$ и $l'+m'$. Эти матрицы хранятся в массивах ХИВЕН и ХОДД

Задача, которую выполняет субритина ПССЛАБ заключается в решении СЛАУ 36, которые за счет свойства 32 матрицы омега разбиваются на две системы, состоящих из $l_{max}(l_{max}+2)$ уравнений.


Далее описание из статьи.

блок схему алгоритма привести
\end{comment}

\section{Модификации исходного кода (в разработке)}


\subsection{erf}




