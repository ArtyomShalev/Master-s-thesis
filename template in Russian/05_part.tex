\chapter{\textbf{Исследование эффекта углового закрепления связанных состояний в континууме в двумерных периодических структурах}}

\section{Введение}
Усиление ЭМ волн является одной из основных задач, решаемых в оптике, радиофизике, фотонике, плазмонике и других схожих областях. Одним из способов усиления ЭМ поля является создание структур, позволяющих локализовать свет в интересующей области. Для этих задач хорошо подходят высокодобротные системы, выполняющие роль резонатора. Есть несколько спобов достичь высокой добротности резонатора:
\begin{itemize}
	\item за счет увеличения размера резонатора;
	\item за счет использования фотонного кристалла, которые можно использовать в качестве нанорезонатора, или планарного фотонного кристалла, который будет выступать мембраной в резонаторе (ссылка на работу из иоффе);
	\item за счет использования эффекта полного внутреннего отражения в резонаторах с модами шепчущей галереи(3-5 из одита выбрать);
	\item использование относительно новых физических явлений, которые могут значительно увеличить добротность системы, к которым относятся ССК.  
\end{itemize}

Сами по себе ССК являются математическим объектом, который обладает бесконечной добротностью и, соответственно, крайне узким резонансом. К тому же, ССК можно наблюдать только в идеальных бесконечных структурах без потерь [ссылки 22-24 из статьи С.Лепешова]. На практике в реальных структурах рассматриваются квази-ССК, которые обладают конечной добротностью.

Так, например, локализация света в субволновых фотонных структурах возможна на основе диэлектрических частиц с высоким показателем преломления (6-8 одит), в которых будут наблюдаться ССК. Также на сегодняшний день ССК были обнаружены в фотонных кристаллах(16,17), массивах волноводов(18, статья просидинг) и метаповерхностях(19). Данный физический феномен находит различные применения при создании лазеров(20,21), биосенсоров(22), нелинейных преобразователей частоты(10) и других оптических приборов, принцип работы которых основан на локализации света \cite{2021_odit_supercavity_modes}.

\section{Связанные состояния в континууме (все ссылки не надо, только самые знакомые работы)}

ССК впервые было теоретически предсказаны в 1929 году в работе по квантовой механике Ю.Вигнера и Д.фон Неймана~\cite{von1993merkwurdige}, ими было обнаружено, что помимо связанных состояний, которые обладают дискретным энергетическим спектром, и свободных состояний с непрерывным спектром существуют также связанные состояния, энергетический спектр которых является непрерывным. Такие состояния не могут излучать энергию в свободное пространство, за счет чего добротность таких состояний является бесконечной. В силу волновой природы, данное явление наблюдается не только в квантовой механике, но также и в других научных областях, включая фотонику.

\begin{figure}[h]
	\begin{center}
		\includegraphics[width=0.9\textwidth]{images/bic_concept.png}
		\caption{bic concept}
		\label{fig:bic_concept}
	\end{center}
\end{figure}

В фотонике ССК были экспериментально обнаружены в 2013 году~\cite{2013_hsu_nature} (дать описание)  ()адаптированная илюстрация).

Таблица классификации по механизмам

Ниже будет приведено более подробное описание типов ССК, представленных в таблице. 

\subsection{Защищенные симметрией ССК}
Когда система симметрична относительно отражения или поворота, тогда моды, характерные различным классам симметрии, являются полностью развязанными. В таком случае достаточно просто найти связанное состояние одного класса симметрии в непрерывном спектре (континууме) другого. Такие моды остаются развязанными до тех пор, пока сохраняется симметрия.

Защищенные симметрией ССК достаточно просто продемонстрировать на примере расппространения звука в акустическом волноводе и плоской пластины, помещенной посередине (рисунок). Давление воздуха $P$ удовлетворяет скалярному уравнению Гельмгольца с граничными условиями Неймана на поверхности стенок волновода и пластины. Волновод поддерживает распространение волн в направлении $x$. При этом в направлении $y$ моды волновода разделяются на четные и нечетные относительно центра волновода. Нечетная мода имеет критическую частоту $\frac{\pi c_s}{h}$, где  $c_s$ -- скорость звука, а $h$ -- ширина волновода. Пластина заркально симметрична и, соответственно, моды, локализованные вблизи пластины, тоже разделяются на четные и нечетные.

доописать про акустический волновод

Похожие защищенные симметрией ССК наблюдаются в поверхностных волнах на воде~\cite{2013_dyn_loc_phen}, кватовых нитях~\cite{2009_bic_in_quantum_systems}. Защищенные симметрией ССК наблюдаются в $\Gamma$-точке двумерных периодических структур, например, в планарных фотонных кристаллах~\cite{2014_top_nature_bic, 2019_zarina_multipolar_bic}.


\subsubsection{Разделяемые ССК}
ССК могут возникать при решении задачи на собственные числа методом разделения переменных. Гамильтониан двумерной системы представляется в виде $H=H_x(x)+H_y(y)$, где $H_x$ воздействует только на переменную $x$, а $H_y$ -- на $y$. В ходе решения одномерных задач на собственные числа $H_x\psi_x^{(n)}(x)=E_x^{(n)}\psi_x^{(n)}(x)$ и $H_y\psi_y^{(m)}(y)=E_y^{(m)}\psi_y^{(m)}(y)$
волновая функция представляется в виде $\psi = \psi_x^{(n)}(x)\psi_y^{(m)}(y)$, где оба множителя соответствуют связанным состояниям, при этом их суммарная энергия лежит в непрерывном спектре. Такой тип ССК был впервые предложен Робником (66 ссылка) и изучался в различных квантовых системах(67-69) и решении уравнений Максвелла для двумерных структур. Данный вид ССК пока предсказан только теоретически, но существуют гипотезы о системах, в которых может быть получено экспериментальное подтверждение данному типу ССК. К таким системам относятся оптические ловушки для холодных атомов и решетки, которые могут быть описаны приближением сильно связанных электронов (74).

\subsubsection{ССК, возникающие при изменении параметров}
При небольшом числе открытых дифракционных каналов настройки параметров системы может быть достаточно для того, чтобы полностью подавить излучение. Если излучение характеризуется  $n$ степенями свободы, как минимум $n$ параметров должны быть настроены для обнаружения ССК. Подавление излучения можно рассматривать со стороны деструктивной интерференции нескольких изулающих компонент. Здесь возможны три различных сценария, описанных ниже.

\subsubsection{Фабри-Перо ССК} 
Данная система похожа на резонатор Фабри-Перо, в котором в качестве стенок выступают две резонансные структуры. Резонасные структуры подбираются таким образом, чтобы компенсировать любое излучение вне данной системы и локализовать свет внутри. ССК образуются, когда расстояние между структурами $d$ подбирается таким образом, чтобы набег фазы, который приобретает свет при прохождении от одной стенки к другой был кратен $2\pi$. 

Фабри-Перо ССК обычно наблюдаются в системах с двумя идентичными резонансами, настроенными на частоту излучения, например, пара квантовых точек, соединенных квантовой нитью (89-93 посмотреть публикации), две металлические цепочки на металлической подложке (94 посмотреть публикации), в фотонике 96-98, 99, 100, 101, 102, в акустических волноводах (103). Уникальной особенностью данных ССК является то, что несколько резонаторов могут иметь сильное взаимодействие через излучение даже при достаточном отдалении друг от друга. 

\subsubsection{Параметрические ССК на одиночном резонансе} 
Такие виды ССК в англоязычной терминологии еще называются случайными. Возникают тогда, когда одну моду можно представить в виде суперпозиции других мод (19 вики), каждая из которых меняется в зависимости от параметров структуры. В какой-то момент происходит деструктивная интерференция и наблюдается ССК данного типа.

Такие ССК в основном рассматриваются в планарных фотонных кристаллах[41 вики]. Помимо фотонных периодических структур[43][44], ССК встречаются для волн на воде[45], квантовых волноводах[46], механических резонаторах[47], и в виде поверхностных волн[48]. (все вики) Такие ССК наблюдаются в линейных периодических структурах прямоугольников (134, 135), цилиндров (136) и сфер (137).

Данные ССК не гарантируются симметрией, они являются устойчивыми к небольшим изменениям параметров системы. Также такие ССК характерны не только для периодических структур, что было теоретически показано в работах по акустическим волноводам, ссылка какая-то из 143-148), квантовым волноводам с примесями (149-151) и механическим резонаторам (154).

\subsubsection{ССК Фридриха-Винтгена} 
В Фабри-Перо ССК две резонансные структуры разнесены друг от друга в отличие от ССК Фридриха-Винтгена, которые возникают в одной резонасной структуре за счет деструктивной интерференции, которая может наблюдаться при излучении разных мод в один и тот же радиационный канал. В общем случае возникновение таких ССК возможно, когда число резонансов превосходит число радиационных каналов, но при этом растет и число "подстроечных" параметров~\cite{2003_discrete_charging_quantum_dots}.

ССК Фридриха — Винтгена рассматривались в атоме водорода в магнитном поле[26] и экспериментально проявлялись в подавлении автоионизации атома бария[27], топологических изоляторах с дефектами[28] и многих других квантовых системах[29][30]. Также этот тип ССК встречается в двумерных и трехмерных цилиндрических открытых акустических резонаторах[31][32]. В фотонике случайные ССК в периодических структурах могут иногда появляться как ССК Фридриха-Винтгена[33][34]. Примечательным примером является теоретическая реализация ССК на слоистой сферической наночастице из специфически подобранных материалов[36][37], при этом ССК появляются за счет взаимодействия различных дипольных мод одной и той же частицы. По тому же механизму могут быть реализованы высокодобротные состояния и в несферических диэлектрических резонаторах[38][39], но в этом случае моды состоят из бесконечного числа мультиполей[40], а добротность не уходит полностью в бесконечность.





\begin{comment}
\subsection{Мультипольный анализ ССК}
Для обнаружения и анализа ССК в современной науке используется множество методов:
mode expansion method[34 Zarina]
resonance state expansion method[35,36]
итд по Зарине

В последние несколько лет в нанофотонике электромагнитная мультипольная теория (уточнить термин) является доминирующим подходом при анализе резонансов системы (в таком виде совсем не нравится предложение)
Основным преимуществом метода мультипольного разложения является представление произвольного распределения поля в виде суперпозиции полей, образованных определенным набором мультиполей [60,61 Zarina]. Такой метод широко применяется при определении поляризации и диаграмме направленности рассеянного излучения от плазмонных и диэлектрических структур. Это может применяться для поляризаторов[61], диэлектрических наноантеннах[65] и где то еще.

С помощью ММР были объяснены некоторые недавно обнаруженные оптические эффекты - анапольное состояние и эффект Керкера (ссылки).

Далее по статье Зарины как можно использовать ММР для описания обоих типов ССК.
\end{comment}

\section{Численное исследование углового закрепления ССК}

Численное исследование эффекта углового закрепления осуществлялось в одномультипольном приближении в программе MULTEM. Одномультипольное приближение заключается в том, что в качестве двумерной периодической структуры выступает не привычная решетка, состоящая из реальной частицы определенной формы (сферы в случае MULTEM), а решетка, в которой в каждой элементарной ячейке размещен точечный мультиполь с определенным порядком ($l$) и ориентацией ($m$) (рис. 
\ref{fig:multipolar_lattice_with_spheres}).

\subsection{Мультипольная решетка}
В программе MULTEM данная система является все тем же массивом сфер, упакованных в квадратную решетку, обладающих только одним мультипольным откликом. Важным также является то, что мультиполи, из которых состоит решетка, должны быть одиноковой ориентации, при соблюдении указанных требований такую решетку можно назвать мультипольной. В данной работе рассматривается мультипольная решетка, состоящая из магнитных октополей, ориентированных вдоль оси $z$ (рис \ref{fig:bic_pinning1}а), в терминах квантовых чисел это магнитный мультиполь с $l = 3$ и $m = 0$. 

\begin{figure}[h]
	\begin{center}
		\includegraphics[width=0.9\textwidth]{images/multipolar_lattice_with_spheres.jpg}
		\caption{multipolar lattice with spheres}
		\label{fig:multipolar_lattice_with_spheres}
	\end{center}
\end{figure}

Можно предположить, что мультипольная решетка будет излучать также как и одиночный мультиполь. То есть взаимодействие мультиполей в решетке будет влиять на перераспределение интенсивности излучения, но не изменит диаграмму направленности (дать ссылки из статьи).

\subsection{Угловое закрепление незащищенных симметрией ССК}
При рассмотрении диаграммы направленности магнитного октополя видно, что он имеет ярковыраженный минимум в пределах $\theta \in [0,\pi/2]$. Угол, при котором наблюдается минимум обозначим $\theta_0$, тогда $\theta_0$ можно вычислить аналитически при нахождении экстремума выражения которое определяет диаграмму направленности магнитного октополя (см. приложение ссылка на прилолжение). Решение уравнения $P'_3(\cos\theta)=0$ дает значение $\theta_0 = 63.4^\circ$, что соответствует положению незащищенного симметрией ССК. На рисунке \ref{fig:bic_pinning1}г схематично продемонстрирован процесс образования незащищенных симметрией ССК в мультипольных решетках. Синяя линия является дисперсионной кривой мультипольной решетки и определяется резонасным взаимодействием решетки с падающей на нее волной. Красная пунктирная линия характеризует нодальную линию мультиполя -- линию, вдоль которой отсутствует излучение. На пересечении этих двух кривых образуется незащищенное симметрией ССК, что и продемонстировано на карте коэффициента отражения мультипольной решетки (рис. \ref{fig:bic_pinning1}в).

 
\begin{figure}[h]
	\begin{center}
		\includegraphics[width=0.9\textwidth]{images/bic_pinning1.jpg}
		\caption{bic pinning 1}
		\label{fig:bic_pinning1}
	\end{center}
\end{figure}
 
Крайне важным здесь является то, что дисперсионная кривая зависит от параметров системы, в частности периода решетки, в отличие от нодальной линии. В связи с этим для любой мультипольной решетки можно аналитически определить положение незащищенного симметрией ССК, которое к тому же будет сохранять свое положение при изменении периода решетки. На рисунке \ref{fig:bic_pinning2} представлены семейства карт коэффициента отражения при разных отношения периода решетки к диаметру частицы. Видно, что положение незащищенного симметрией ССК остается неизменным. 

Так как рассматриваемая система является субволновой, то она имеет только один дифракционный канал. При увеличении периода решетки дисперсионная кривая поднимается вверх по частоте и видно, что при достижении порога дифракции незащищенный симметрией ССК практически исчезает (рис. \ref{fig:bic_pinning2}г), так как открываются дополнительные дифракционные каналы, через которые может происходить излучение. 

\begin{figure}[h]
	\begin{center}
		\includegraphics[width=0.9\textwidth]{images/bic_pinning2.jpg}
		\caption{bic pinning 2}
		\label{fig:bic_pinning2}
	\end{center}
\end{figure}


\section{Выводы по главе}