\chapter{\textbf{Сравнение CST Microwave Studio и MULTEM}}

\section{Введение}
Коммерческая основа CST Microwave Studio (далее CST) накладывает на нее определенные обязательства к точности и воспроизводимости результатов моделирования, а коммерческий успех данного продукта позволяет сделать вывод о том, что его результатам можно доверять.

Что же касается программ с открытым исходным кодом, к которым относится MUTLEM, то, очевидно, что разработчики редко могут гарантировать определенный результат, поэтому к использованию таких программ в научных или промышленных исследованиях стоит подходить с осторожностью. В связи с этим перед использованием MUTLEM в расчетах необходимо верифицировать правильность его работы. 

В данной главе будет проведно сравнение этих двух программных продуктов по скорости и точности выполняемых ими вычислений на примере двумерной бесконечной периодической структуры.

\section{Описание исследуемой системы}
В качестве исследуемой системы выступала двумерная бесконечная периодическая структура с квадратной решеткой. В каждый узел решетки помещалась диэлектрическая сфера с относительной диэлектрической проницаемостью $\varepsilon = 50$, диаметр сферы $d = 20$ мм, а период решетки $a = 40$ мм. Система облучалась плоской ТЕ поляризованной волной под углом $\theta=45^{\circ}$ к нормали структуры. Внешний вид исследуемой системы представлен на рисунке \ref{fig:system_concept}.

\begin{figure}[h]
	\begin{center}
		\includegraphics[width=0.9\textwidth]{images/system_concept.jpg}
		\caption{Внешний вид исследумой системы}
		\label{fig:system_concept}
	\end{center}
\end{figure}

\section{Аппроксимация спектра}
Спектры, полученные MULTEM и CST Microwave Studio были сняты в микроволновом диапазоне 0.13 -- 0.155 м. Спектр коэффициента отражения представлен на рисунке \ref{fig:fano_fit}. В качестве параметра, по которому производилось сравнение было выбрано положение резонанса $\lambda_0$.

Для извлечения $\lambda_0$ была проведена аппроксимация при помощи программного кода, написанного на языке программирования Python. Полученный спектр аппроксимировался формулой резонанса Фано, предложенной в работе ~\cite{FanoKK}. Резонанс Фано в зависимости от длины волны описывается формулой:
\begin{equation*}
	f(\lambda) =  \frac{I(q+\lambda - \lambda_0)^2}{1+q^2(1+(\lambda - \lambda_0)^2)}
\end{equation*}

Данная аппроксимация достаточно хорошо описывает спектр, полученный численным расчетом, что продемонстрировано на рисунке \ref{fig:fano_fit}.
\begin{figure}[h]
	\begin{center}
		\includegraphics[width=0.9\textwidth]{images/fano_fit.png}
		\caption{Аппроксимация спектра формулой резонанса Фано}
		\label{fig:fano_fit}
	\end{center}
\end{figure}
Процедура аппроксимации позволяет получить точное значение $\lambda_0$, что является необходимым для сравнения результатов моделирования разных программ.

\section{Моделирование в MUTLEM}

Конфигурационный файл, который был использован для задания параметров моделирования приведен (ссылка на приложение).
Параметр $l_{max}$ определяет максимальный порядок ВСГ, которые учитываются при расчете. Чем этот параметр выше, тем выше точность расчета, однако при этом значительно возрастает время расчета. Максимальное значение параметра $l_{max} = 7$, тем не менее для некоторых задач достаточно и меньших значений параметра, так как дальнейший вклад ВСГ в результаты расчета становится незначительным. На рисунке \ref{fig:lmax} представлена зависимость положения резонанса от всех значений $l_{max}$, которые позволяет выставить MULTEM.

\begin{figure}[h]
	\begin{center}
		\includegraphics[width=0.9\textwidth]{images/lmax.png}
		\caption{Зависимость положения резонанса от количества рассматриваемых ВСГ (рисунок подправить)}
		\label{fig:lmax}
	\end{center}
\end{figure}

При $l_{max} \geq 3$ видно, что дальнейшее изменение $\lambda_0 < 10^{-6}$ \%. В связи с этим было решено зафиксировать значение $l_{max} = 3$ и использовать его при дальнейшем сравнении с результатами расчета CST Microwave Studio. При $l_{max} = 3$ $\lambda_0 = 0.14274$ м.

\section{Моделирование в CST Microwave Studio}
При моделировании исследуемой системы в CST был использован расчет в частотной области (Frequency Domain Solver), так как данный способ позволяет считать все порты одновременно и независимо друг от друга и является отличным инструментом для моделирования периодических структур. Для моделирования двумерных бесконечных периодических структур используются специальные граничные условия "Unit cell"\,, которые продлевают элементарную ячейку в двух направлениях до бесконечности.

\begin{figure}[h]
	\begin{center}
		\includegraphics[width=0.9\textwidth, height = 7 cm]{images/cst_model.jpg}
		\caption{Исследуемая модель в CST}
		\label{fig:cst_model}
	\end{center}
\end{figure}

\begin{figure}
	\begin{center}
		\includegraphics[width=0.9\textwidth]{images/cst_vs_multem.png}
		\caption{cst vs multem}
		\label{fig:cst_vs_multem}
	\end{center}
\end{figure}

Плотность сетки регулировалась определением числа тетраедров на всю расчетную область. В CST плотность сетки задается во вкладке "Global properties" (рис. \ref{fig:mesh_config}) Определяется минимальное и максимальное количество ячеек на одну длину волны как для модели, так и для среды, в которую модель помещена. Также при расчетах использовалась адаптивная сетка, поэтому количество тетраедров, используемых в расчете выше, чем при ручном определении сетки (\ref{fig:mesh}). 

\begin{figure}[h]
	\centering
	\begin{subfigure}{.5\textwidth}
		\centering
		\includegraphics[width=.7\linewidth]{images/mesh_config.jpg}
		\caption{A subfigure}
		\label{fig:mesh_config}
	\end{subfigure}%
	\begin{subfigure}{.5\textwidth}
		\centering
		\includegraphics[width=.7\linewidth]{images/mesh.jpg}
		\caption{A subfigure}
		\label{fig:mesh}
	\end{subfigure}
	\caption{A figure with two subfigures}
	\label{fig:cst_mesh}
\end{figure}




На рисунке \ref{fig:cst_vs_multem} представлена зависимость относительной ошибки расчета $\lambda_0$, спектры отражения для которой были расчитаны при помощи CST, от числа тетраедров.

Данная зависимость демонстрирует, что результат расчета CST сходится к результату расчета MULTEM, так как рассчитанные точки в двойном логарифмическом масштабе укладываются на прямую с коэффициентом детерминации $R^2 = 0.9992$. Несмотря на то, что нельзя утвержать, что при дальнейшем увеличении количества тетраедров тренд также останется близким к линейному, данная зависимость демонстрирует, что при самом длительном расчете в CST ($7200$ с) относительная ошибка $\delta\lambda_0 = 3.2 \cdot 10^{-4}$ \%. В то время, как время расчета в MULTEM составило (столько то сек). В таблице \ref{cst_vs_multem_tab} приведены данные по относительной ошибке для двух рассмотренных инструментов.  


\begin{equation*}
	\delta\lambda_0 = \frac{|\lambda_0^{CST}-\lambda_0^{MULTEM}|}{\lambda_0^{MULTEM}} \cdot 100\%
\end{equation*}

\begin{table}[]
	\centering
	\caption{cst vs multem tab}\label{cst_vs_multem_tab}
	\begin{tabular}{|c|c|c|c|}
		\hline
		\multicolumn{2}{|c|}{\textbf{Время расчета, с}} & \multicolumn{2}{c|}{\textbf{Ошибка, \%}} \\ \hline
	MULTEM	& CST &  MULTEM                 & CST \\
	\hline
		\multirow{6}{*}{3} & 140 & \multirow{6}{*}{$<3.2 \cdot 10^{-4}$} & $9.2 \cdot 10^{-2}$ \\ 
		& 246 &                  & $2.7 \cdot 10^{-2}$ \\  
		& 502 &                   &  $8.6 \cdot 10^{-3}$\\
		& 1304 &                   & $2.2 \cdot 10^{-3}$ \\
		& 3675 &                   &  $8.3 \cdot 10^{-4}$\\
		& 7200 &                   & $3.2 \cdot 10^{-4}$ \\ \hline	
	\end{tabular}
	
\end{table}


\section{Выводы по главе}


