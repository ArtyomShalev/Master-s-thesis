\chapter{\textbf{Программная реализация режима мультипольного разложения}}
\section{Введение}
Мультипольный подход является мощным инструментом при анализе взаимодействия электромагнитного поля с веществом. Основное преимущество мультипольного подхода состоит в том, что он обеспечивает представление произвольного распределения поля в виде суперпозиции полей, созданных набором одиночных мультиполей. Анализ вкладов различных мультиполей может помочь, например, при решении обратной задачи.

Также мультипольный подход является ключевым при построении многих распространенных численных алгоритмов решения задач рассеяния на объектах произвольной формы, к которым относят метод Т-матриц, который рассмотрен в разделе \ref{sec:T_matrix} и метод матриц рассеяния (ссылки из статьи фитцпатрика 28, 27).
   
Однако в оригинальной версии MULTEM отсутствовала возможность применять мультипольное разложение, MULTEM в расчете учитывал мультиполи всех порядков и ориентаций вплоть до заданного значения $l_{max}$. В данной главе будет описана математическая модель одномультипольного приближения и ее программная реализация на языках программирования Fortran и Python с применением конфигурационного ini-файла. 

\section{Математическая модель мультипольного разложения}
Для сферы в силу симметрии Т-матрица является диагональной и вид выражений (\eqref{Tmatrix1}--\eqref{Tmatrix3}) упрощается. Так как MULTEM работает только со сферическими рассеятелями, то в его работе используются только две диагональные матрицы вида:

\begin{equation*}
	T_{ee(hh)} = \begin{pmatrix}
		T_{ee(hh)}^{11} & 0 & ... 	& 0\\
		0 & T_{ee(hh)}^{22} & ... 	& 0\\
		...			&	...	& ... 	& 0\\
		0			&	0	&  0 	& T_{ee(hh)}^{l_{max}l_{max}}
	\end{pmatrix}
\end{equation*}	

При расчетах в MULTEM Т-матрицы используются только в одном месте при решении системы линейных алгебраических уравнений вида:

\begin{equation*}
	 \begin{pmatrix}
		I - T_{ee}\Omega_{ee} 	& 	T_{ee}\Omega_{eh} \\
		T_{hh}\Omega_{he}		&	I - T_{hh}\Omega_{hh} \\
	\end{pmatrix} \cdot 
\begin{pmatrix}
	b_{+e} \\
	b_{+h}	\
\end{pmatrix} = 
\begin{pmatrix}
		T_{ee}a_{0e} \\
		T_{hh}a_{0h}	\\
\end{pmatrix}
\end{equation*}	

где то, то и то

Таким образом, путем искуственного зануления коэффициентов Т-матрицы, соответсвующих определенному порядку мультиполя, можно исключить его влияние на общую картину рассеяния. Например, при $T_{ee+}^{11} = 0$ полностью исключается вклад электрического диполя. 

Для того, чтобы управлять не только порядками мультиполей, но и их ориентацией необходимо занулить все коэффициенты рассеянного поля, не соответствующие заданным числам $l$ и $m$, которые определяет пользователь перед запуском расчета.
	

\section{Реализация режима мультипольного разложения}
\begin{comment}
Режим мультипольного разложения можно реализовать путем модернизации элементов Т-матрицы.  рассматривалась теория Ми и  понятие Т-матрицы. 
Лучше в нотации Борена Хафмана

$\begin{pmatrix}	f_{mn}  \\
					g_{mn}   \end{pmatrix} = 
T\begin{pmatrix}	a_{mn}  \\
					b_{mn}   \end{pmatrix}$

Более подробный вид матрицы


Для реализации режима мультипольного разложения необходимо оставить в Т-матрицы только те элементы, которые отвечают определенным порядкам ВСГ, при этом все остальные элементы обнуляются. В таком случае в картине рассеяния будут наблюдаться только заданные пользователем мультипольные отклики.

Т-матрицы в программном коде MULTEM используются в субрутинах PCSLAB и SETUP (ссылка на 2 главу).

Про конкретный алгоритм, зайти в код - посмотреть что зануля
возможно добавить программную реализацию
\end{comment}
Описать программную реализацию, структуру конфиг файлов

\section{Верификация результатов}

После программной реализации режима мультипольного разложения была проведена верификация на основе данных из работы (ссылка). В указанной работе был произведен мультипольный анализ двумерной периодической структуры, состоящей из золотых сфер, упакованных в гексагональную решетку. В работе были рассмотрены влияния различных мультипольных вкладов, которые по-разному ведут себя в зависимости от угла падения ЭМ волны с фиксированной частотой. В таблице \ref{multipoles} приведена расшифровка мультиполей, которые приведены на рисунке \ref{fig:mul_exp_verification}.

\begin{table}[h]
	\centering
	\caption{multipoles - ссылка на Рокштуля}\label{multipoles}
	\begin{tabular}{|c c c c|}
		\hline
		& $\mathbf{l}$ & $\mathbf{m}$     & \textbf{тип}           \\ \hline
		$p_s$ & 1 & -1,1  & электрический \\
		$m_z$ & 1 & 0     & магнитный     \\ 
		$q_{ks}$ & 2 & -2, 2 & электрический \\ \hline
	\end{tabular}
\end{table} 

\begin{figure}[h!]
	\begin{center}
		\includegraphics[width=0.9\textwidth]{images/verification_design.jpg}
		\caption{verification design}
		\label{fig:verification_design}
	\end{center}
\end{figure}


\begin{figure}[h!]
	\begin{center}
		\includegraphics[width=0.9\textwidth]{images/mul_exp_verification.jpg}
		\caption{mul exp ver}
		\label{fig:mul_exp_verification}
	\end{center}
\end{figure}

Из приведенного сравнения видно, что результаты совпадают, что позволяет сделать вывод о достоверности результатов, которые получены при помощи MULTEM в режиме мультипольного разложения.

\section{Выводы по главе}